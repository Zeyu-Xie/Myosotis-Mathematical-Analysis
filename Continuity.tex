
\chapter{函数极限与连续性}

\section{函数的极限}

\subsection{函数极限定义}

\begin{definition}[函数极限]
  $x_0 \in \mathbb{R}$,$f$在$x_0$邻域有定义($x_0$处可以没有),$\forall \epsilon, \exists \delta, ~s.t.~ \forall x \in [x_0 - \delta, x_0 + \delta] - \{x_0\}$有$|f(x) - A| < \epsilon$,
  则称$A$为$f(x)$在$x_0$处极限。
\end{definition}

~

\begin{exercise}[Riemann函数]
  Riemann函数定义如下,证明对$\forall x_0 \in [0,1]$,有$\lim \limits _{x \rightarrow x_0}R(x) = 0$
  \begin{equation*}
    R(x) =
    \begin{cases}
      \frac{1}{q}, & x = \frac{p}{q}(\frac{p}{q}\text{是既约真分数})\\
      0, & x\text{是(0,1)中的无理数或者0,1本身}
    \end{cases}
  \end{equation*}
\end{exercise}

\begin{proof}
  $\forall \epsilon$,满足$\frac{1}{q} \geq \epsilon$的正整数$q$只有有限个,
  从而$\frac{p}{q}$有限,令这些数为$x_1,\cdots,x_k$,
  取
  \begin{equation*}
    \delta = \min\{|x_i - x_0|: i = 1,2,\cdots,k\}
  \end{equation*}
  当$x$属于$x_0$的$\delta$邻域时,显然$R(x) < \epsilon$,因此$\lim \limits _{x \rightarrow x_0}R(x) = 0$
\end{proof}

\begin{note}
  上述练习说明Riemann函数在每个点极限为$0$,
  在$(0,1)$的无理点和$0,1$连续,
  在$(0,1)$有理点间断。
  这说明Riemann函数是几乎处处连续的,根据实变可知其Riemann可积。
\end{note}


\begin{theorem}[极限的唯一性]
  若函数的极限存在,则它是唯一的。
\end{theorem}

\begin{proof}
  反设有两个极限,取$\epsilon$为差距的一半,根据定义可证。
\end{proof}


\subsection{函数极限的性质}


\begin{theorem}[函数极限四则运算]
  若$f(x), g(x)$在$x_0$处极限存在,
  $f+g, f-g, f \times g, \frac{f}{g}$极限存在且极限为对应极限的运算(除法要求分母极限非0)。
\end{theorem}

\begin{exercise}[函数绝对值的极限]
  设$\lim \limits _{x \rightarrow x_0}f(x) = A$,则$\lim \limits _{x \rightarrow x_0}|f(x)| = |A|$,
  反之不一定成立。
\end{exercise}

\begin{proof}
  由于$\lim \limits _{x \rightarrow x_0}f(x) = A$,
  $\forall \epsilon, \exists \delta, \forall x \in U(x_0,\delta)$有$|f(x) - A| < \epsilon$,
  而$\left| |f(x)| - |A| \right| \leq |f(x) - A|$,因此极限存在。
  反之有反例$f(x) = \text{sgn}(x)$
\end{proof}

\begin{theorem}[归结原则] \label{thm:数列极限与函数极限}
  $\lim \limits _{x \rightarrow x_0} f(x) = A$当且仅当任意收敛到$x_0$的数列$x_n$,有$\lim \limits _{n \rightarrow \infty} f(x_n) = A$。
\end{theorem}

\begin{proof}
  (1)左推右:$\forall \epsilon, \exists \delta, |x - x_0| < \delta$时$|f(x) - A| < \epsilon$,
  对于$x_n$,由于$\lim \limits _{n \rightarrow \infty} x_n = x_0$,
  $\exists N, \forall n > N, |x_n - x_0| < \delta$,
  因此$|f(x_n) - A|< \epsilon$

  (2)右推左:反设$\lim \limits _{x \rightarrow x_0}f(x) \neq A$,则$\exists \epsilon_0, \forall \delta,\exists x \in U(x_0,\delta)$满足
  $|f(x) - A| > \epsilon_0$,
  每次取$\delta^{\prime} = \delta,\frac{\delta}{2},\cdots,\frac{\delta}{n}$,
  每个$\delta^{\prime}$邻域对应一个$x_n$,但是总有$|f(x_n) - A| > \epsilon_0$,
  这与$\lim \limits _{n \rightarrow \infty} f(x_n) = A$矛盾。
\end{proof}

~

\begin{exercise}[用归结原则证明不收敛]
  证明:$\lim \limits _{x \rightarrow 0} \sin \frac{1}{x}$不存在。
\end{exercise}

\begin{proof}
  令$x_n = \frac{1}{(2n + \frac{1}{2}) \pi}, x_n^{\prime} = \frac{1}{2n \pi}$,
  两者均收敛至$0$,但是$f(x_n) \rightarrow 1$,$f(x_n^{\prime}) \rightarrow 0$,
  从而无极限。
\end{proof}




\subsection{无穷大与无穷小}

\begin{definition}[O与o]
  $f,g$在$x_0$附近有定义,$g(x) \neq 0$,
  若$\lim \limits _{x \rightarrow x_0} \frac{|f(x)|}{|g(x)|} \leq M$,则记$f(x) = O(g(x))$;
  若$\lim \limits _{x \rightarrow x_0} \frac{|f(x)|}{|g(x)|} = 0$,
  则记$f(x) = o(g(x))$
\end{definition}

\begin{definition}[$\Omega$]
  $f,g$是两个算法,若$\exists C \in \mathbb{R}^+$使得$n > n_0$时$0 \leq Cg(n) \leq f(n)$,则记$f(n) = \Omega(g(n))$
\end{definition}

\begin{definition}[$\Theta$]
  若$f(n)$是$\Omega(g(n))$且$O(g(n))$的,
  则记$f(n) = \Theta(g(n))$
\end{definition}

\begin{note}
  记忆方法:全以多项式为例,$O$表示阶数(次数)大于等于,$o$表示阶数大于,$\Omega$表示阶数小于等于,$\Theta$表示阶数等于。
\end{note}


\subsection{使用定义直接计算}

\begin{exercise}[几个经典函数极限]
  (1)计算$\lim \limits _{n \rightarrow \infty} \sin^2(\pi \sqrt{n^2 + n})$

  (2)$\lim \limits _{x \rightarrow +\infty}\frac{[xf(x)]}{x}$,其中$\lim \limits _{x \rightarrow +\infty}f(x) = a$
\end{exercise}

\begin{solution}
  (1)$\sin^2(\pi \sqrt{n^2 + n}) = \sin^2 (\pi \sqrt{n^2 + n} - \pi n) = \sin^2 (\frac{n\pi}{\sqrt{n^2 + n} + n})$,
  极限为$\sin^2(\frac{\pi}{2}) = 1$

  (2)由于$xf(x) - 1 < [xf(x)] \leq xf(x)$,
  因此$f(x) - \frac{1}{x} < \frac{[xf(x)]}{x} \leq f(x)$,
  显然$\lim \limits _{x \rightarrow +\infty} \left( f(x) - \frac{1}{x} \right) = \lim \limits _{x \rightarrow +\infty}f(x) = a$,
  因此由夹逼准则,极限为$a$
\end{solution}

~

\begin{exercise}[几个经典递推问题]
  (1)$f(x)$在$(0,+\infty)$单调增,$\lim \limits _{x \rightarrow +\infty}\frac{f(2x)}{f(x)} = 1$,
  证明$\forall a > 0$有$\lim \limits _{x \rightarrow +\infty}\frac{f(ax)}{f(x)} = 1$
\end{exercise}

\begin{proof}
  
\end{proof}



\section{函数极限的计算}

\subsection{等价无穷小公式}

\begin{equation*}
  \begin{array}{llll}
    \text{三角:}&\sin x \sim x&1 - \cos x \sim \frac{1}{2}x^2&\tan x \sim x\\
    \text{反三角:}& \arcsin x \sim x& \arctan x \sim x&\\
    \text{指数、对数:}&\ln(1 + x)\sim x& e^x - 1 \sim x& a^x - 1 \sim x \ln a\\
    \text{其他:}&(1 + x)^{\alpha} - 1 \sim \alpha x&&
  \end{array}
\end{equation*}


\subsection{Taylor展开公式}

\begin{equation*}
  \begin{array}{lll}
    \text{指数、对数:}& e^x = \sum\limits_{n = 0 }^{\infty}\frac{x^n}{n!}& \ln(1 + x) = \sum\limits_{n = 1}^{\infty}(-1)^{n-1}\frac{x^n}{n} \\
                       \text{三角:}&\sin x = \sum\limits_{n = 0}^{\infty}(-1)^n \frac{x^{2n+1}}{(2n+1)!}& \cos x = \sum\limits_{n = 0}^{\infty}(-1)^n \frac{x^{2n}}{(2n)!} \\
    \text{反三角:}&\arcsin x =\sum\limits_{n = 0}^{\infty}\left( \frac{(2n)!}{2^{2n}(n!)^2} \right)\frac{x^{2n+1}}{2n+1} =  x + \frac{1}{6}x^3 + o(x^3) &\arctan x = \sum\limits_{n = 0}^{\infty} \frac{(-1)^nx^{2n+1}}{2n+1} = x - \frac{1}{3}x^3 + o(x^3) \\
                       \text{分式:}&\frac{1}{1+x} = \sum\limits_{n = 0}^{\infty}(-1)^n x^n&\frac{1}{1 - x} = \sum\limits_{n = 0}^{\infty}x^n \\
                       \text{其他:}&(1+x)^{\alpha} = 1 + \alpha x + \frac{\alpha(\alpha - 1)}{2}x^2 + o(x^2)&(1 + x)^{\alpha} = \sum\limits_{n = 0}^{\infty}{\alpha \choose n} x^n
  \end{array}
\end{equation*}





\section{函数的连续性}

\subsection{函数连续的定义}

\begin{definition}[单点连续]
  $f: [a,b] \rightarrow \mathbb{R}$,$x_0 \in [a,b]$若$\lim \limits _{x \rightarrow x_0} f(x) = f(x_0)$,则称$f(x)$在$x_0$处连续。
\end{definition}

\begin{note}
  有时也可以展开用$\forall \epsilon, \exists \delta$使得$\forall x \in (x_0 - \delta, x_0 + \delta)$满足$|f(x) - f(x_0)| < \epsilon$证明。
  例如证明 Riemann 函数在无理点连续。
\end{note}

\begin{definition}[单点左右连续]
  若$f(x_0^+) = f(x_0)$则称$f(x)$右连续。左连续同理。
\end{definition}

\begin{note}
  注意右连续是指右极限等于当前点的函数值,
  而非左侧极限靠过来等于当前点的函数值!
\end{note}

~

\begin{exercise}[Dirichlet函数的连续性]
  判断下面Dirichlet函数的连续性、单调性、周期性
  \begin{equation*}
    D(x) =
    \begin{cases}
      1, &x \text{为有理数}\\
      0, &x \text{为无理数}
    \end{cases}
  \end{equation*}
\end{exercise}

\begin{solution}
  (1)周期:任意有理数(显然)

  (2)连续性:处处不连续,因为根据有理数和无理数的稠密性,
  总能在小邻域中找到相差$1$的点,因此不连续。
\end{solution}

~

\begin{exercise}[Riemann函数的连续性]
  证明Riemann函数在无理点连续,在有理点间断。
  \begin{equation*}
    R(x) =
    \begin{cases}
      \frac{1}{q}, & x = \frac{p}{q}\text{为既约分数}\\
      0, & x\text{为无理数}
    \end{cases}
  \end{equation*}
\end{exercise}

\begin{proof}
  (1)有理点:设$x_0 = \frac{p}{q}$为有理点,
  $R(x_0) = \frac{1}{q}$,无理点列趋于$x_0$时,极限为$0$,与函数值不同。

  (2)无理点:
  $x_0$是无理点,
  $\forall \epsilon, \frac{1}{q} > \epsilon$的个数有限,
  设这些点为$x_1,\cdots,x_n$,
  取$\delta < \min\{|x_0 - x_i|: i = 1\sim n\}$,
  $x \in (x_0 - \delta, x_0 + \delta) - \{x_0\}$时
  $|f(x_0) - f(x)| < \epsilon$
\end{proof}

~

\begin{exercise}[初等函数的连续性]
  $f,g,h$连续,证明:$\max \{f(x),g(x)\}, \min \{f(x),g(x)\}, |f(x)|$连续
\end{exercise}

\begin{proof}
  (1)最大、最小:根据$\max \{f(x),g(x)\} = \frac{1}{2}[f(x) + g(x) + |f(x) - g(x)|]$

  (2)绝对值:$|f(x)| = \sqrt{|f(x)|^2}$连续
\end{proof}

\subsection{间断点的概念}

\begin{definition}[第一类间断点]
  跳跃点与可去间断点统称为$f(x)$的第一类间断点
  \begin{itemize}
  \item 跳跃点: 若$f(x)$在$x_0$处左右极限$f(x_0^-), f(x_0^+)$存在但不相等,
  \item 可去间断点:若$f(x)$在$x_0$处左右极限存在且相等,即$f(x_0^-) = f(x_0^+)$
  \end{itemize}
\end{definition}

~

\begin{exercise}[第一类间断点的性质]
  (1)若$f(x)$在$[a,b]$至多有第一类间断点,证明:$f(x)$在$[a,b]$有界
\end{exercise}

\begin{proof}
  (1)若没有间断点,则连续,显然有界。
  否则反设无界,则对$M = 1$,存在$x_1 \in [a,b]$使得$|f(x_1)| > 1$,
  $M = 2$存在$x_2 \in [a,b]$使得$|f(x_2)| > 2$,
  以此类推构造$x_n$。
  由于$x_n \in [a,b]$,
  根据致密性定理,存在收敛子列$x_{n_k}$,
  而由于$f(x)$只有第一类间断点,因此$\lim \limits _{k \rightarrow +\infty}f(x_{n_k})$存在,
  但其趋于无穷,矛盾。
\end{proof}

~

\begin{definition}[第二类间断点]
  若$f(x_0^+)$和$f(x_0^-)$至少一个不存在或者不是有限的数,
  则称$x_0$是$f(x)$的第二类间断点。
\end{definition}

~

\begin{exercise}[判断间断点类型]
  (1)已知$f(x) = \text{sgn}(x),g(x )= x - [x]$,讨论$f(g(x))$和$g(f(x))$的间断点类型
\end{exercise}

\begin{solution}
  (1)$f(g(x))$在$x$为整数时等于$0$,在$x$非整数时大于$0$,因此在整数处取可去间断点。
  $g(f(x)) = 0$,因此连续
\end{solution}

~

\begin{exercise}[单调函数的间断点]
  $f(x)$在$x_0$邻域单增,$f(x_0^-),f(x_0^+)$均存在,证明如下性质,即单调函数的间断点都是跳跃间断点
  \begin{equation*}
    f(x_0^-) = \sup \limits _{x \in U_-^o(x_0)}f(x), f(x_0^+) = \inf \limits _{x \in U_+^o(x_0)}f(x)
  \end{equation*}
\end{exercise}

\begin{proof}
  取$x_1 \in U_+^o(x_0)$,则$\forall x \in U_-^o(x_0)$有$f(x) \leq f(x_1)$,
  从而根据确界原理有上确界$A$,
  从而$f(x) \leq A$且$\forall \epsilon, \exists x^{\prime}$满足$f(x^{\prime}) > A - \epsilon$,
  从而$\forall x \in (x_0 - \delta, x_0)$有$|f(x) - A| < \epsilon$
\end{proof}

\begin{note}
  单调函数左右极限均存在是一个会在考题中使用的结论。
\end{note}


~

\begin{exercise}[单调函数右连续]
  $f(x)$是$\mathbb{R}$上单调函数,
  定义$g(x) = f(x + 0)$,
  证明$g(x)$在$\mathbb{R}$每一点右连续
\end{exercise}

\begin{proof}
  只需要证明$\forall \epsilon, \exists \delta$,当$|x - x_0| < \delta$时,满足
  \begin{equation*}
    |g(x) - g(x_0)| < \epsilon
  \end{equation*}
  根据$g(x_0) = \lim \limits _{y \rightarrow x_0^+}f(y)$得到$y \in (x_0,x_0 + \delta)$时
  有$|f(y) - g(x_0)| < \epsilon$,
  令$y \rightarrow x^+$即可得到
  \begin{equation*}
    |g(x) - g(x_0)| < \epsilon
  \end{equation*}
\end{proof}


~

\begin{exercise}[极限的连续性]
  $f(x)$只有可去间断点,$g(x) = \lim \limits _{y \rightarrow x}f(y)$,证明$g(x)$是连续函数
\end{exercise}

\begin{proof}
  可去间断点说明极限存在,
  $g(x_0) = \lim \limits _{y \rightarrow x_0}f(y)$,
  即对$\forall \epsilon, \exists \delta, y \in U^o(x_0,\delta)$时:
  \begin{equation*}
    g(x_0) - \epsilon < f(y) < g(x_0) + \epsilon
  \end{equation*}
  根据$g(x) = \lim \limits _{y \rightarrow x}f(y)$和保不等式性得到
  \begin{equation*}
    g(x_0) - \epsilon \leq g(x) \leq g(x_0) + \epsilon
  \end{equation*}
  因此$g(x)$在$x_0$连续,由$x_0$任意性可知结论成立。
\end{proof}

~

\begin{exercise}[第一类间断点的经典题目]
  $f(x)$在$(a,b)$至多只有第一类间断点,且对$\forall x,y \in (a,b)$有$f \left( \frac{x + y}{2} \right) \leq \frac{f(x) + f(y)}{2}$,证明:$f(x)$在$(a,b)$连续
\end{exercise}

\begin{proof}
  $\forall x_0 \in (a,b), \forall h$满足$x_0 \pm h \in (a,b)$时,
  根据条件可知:
  \begin{equation*}
    f(x_0) \leq \frac{f(x_0 + h) + f(x_0 - h)}{2} \quad f(x_0 + \frac{h}{2}) \leq \frac{f(x_0) + f(x_0 + h)}{2}
  \end{equation*}
  前者取$h \rightarrow 0^+$,后者分别取$h \rightarrow 0^+, h \rightarrow 0^-$则
  \begin{equation*}
    \begin{cases}
      f(x_0) \leq \frac{f(x_0 + 0) + f(x_0 - 0)}{2}\\
      f(x_0 + 0) \leq \frac{f(x_0) + f(x_0 + 0)}{2}\\
      f(x_0 - 0) \leq \frac{f(x_0) + f(x_0 - 0)}{2}
    \end{cases}
  \end{equation*}
  综合得到$f(x_0 - 0) = f(x_0 + 0) = f(x_0)$,因此在任意点连续。
\end{proof}

\subsection{介值性定理}

\begin{theorem}[零点存在定理]
  $f(x)$是$[a,b]$上连续函数,$f(a)f(b) < 0$,则$\exists \xi \in (a,b)$使得$f(\xi) = 0$
\end{theorem}

\begin{proof}
  (1)使用闭区间套定理进行证明:取$a_1 = a, b_1 = b, c_1 = \frac{a_1 + b_1}{2}$,
  根据$f(c_1)$正负性选取$a_2,b_2$,以此类推,由于$\lim \limits _{n \rightarrow \infty} |a_n - b_n| = 0$,
  $f(a_n)f(b_n) < 0$,根据闭区间套$\exists \xi$使得$f^2(\xi) \leq 0$,因此得出$f(\xi) = 0$

  (2)使用有限覆盖定理证明:根据连续函数局部保号性,$\forall x \in [a,b], \exists \delta_x$使得$f(x)$在$(x-\delta_x ,x + \delta_x) \cap [a,b]$中保号,
  取遍$[a,b]$显然是一族开覆盖,根据有限覆盖定理可找到有限覆盖,
  不妨设这有限覆盖相互有重叠,则全部等号,
  这与$f(a)f(b) < 0$矛盾。
\end{proof}

\begin{theorem}[介值性定理]
  $f(x)$是$[a,b]$中的连续函数,$a \leq x_1 < x_2 \leq b$,$f(x_1) \neq f(x_2)$,
  则$f(x)$在$[a,b]$可取遍$f(x_1),f(x_2)$间任意值。
\end{theorem}

~

\begin{exercise}[构造辅助函数]
  (1)$f(x),g(x)$是$[0,1]$上的连续函数,且在$[0,1]$有相同最大值,证明$\exists \xi \in [a,b]$使得$f(\xi) = g(\xi)$

  (2)重点:$f(x)$是$[0,2a]$连续函数,且$f(0) = f(2a)$,证明$\exists \xi \in [0,a]$使得$f(\xi) = f(\xi + a)$

  (3)重点:$f(x)$是$[0,1]$上连续函数,$f(0) = f(1) = 0$,证明$\forall n \in \mathbb{N}^+, \exists \xi \in [0,1]$使得
  $f(\xi + \frac{1}{n} ) = f(\xi)$

  (4)重点:$f(x)$是$[0,1]$上非负连续函数,$f(0) = f(1) = 0$,证明:对$\forall a \in (0,1), \exists \xi \in [0,1]$,使得$f(\xi) = f(\xi + a)$
\end{exercise}

\begin{proof}
  (1)问题在于最大值取值点不一定一样:若最大值对应$f(x_1),g(x_2)$,
  若$x_1 = x_2$,则显然成立。
  若$x_1 \neq x_2$,则取$F(x) = f(x) - g(x)$,$F(x_1) \geq 0, F(x_2) \leq 0$,从而由零点存在定理成立。

  (2)取$F(x) = f(x) - f(x + a)$,$F(0) = f(0) - f(a), F(a) = f(a) - f(2a) = f(a) - f(0)$,
  因此$F(0) = - F(a)$,显然成立。

  (3)取$F(x) = f(x + \frac{1}{n}) - f(x)$,
  $F(0) = f(\frac{1}{n}) - f(0), F(\frac{1}{n}) = f(\frac{2}{n}) - f(\frac{1}{n}),\cdots, F(\frac{n-1}{n}) = f(1) - f(\frac{n-1}{n})$,
  得到$F(0) + F(\frac{1}{n}) + \cdots + F(\frac{n-1}{n}) = f(1) - f(0) = 0$,
  要么这些全部为零,要么有正有负,根据定理可知。

  (4)取$F(x) = f(x) - f(x+a)$,$F(0) = f(0) - f(a) = -f(a), F(1-a) = f(1-a) - f(1) = f(1-a)$,
  由于$f$非负,因此$F(0)F(1-a) \leq 0$。
\end{proof}

~

\begin{theorem}[连续函数平均值定理]
  $f(x)$在$[a,b]$连续,$x_1,\cdots,x_n \in [a,b]$,
  证明:$\exists \xi \in [a,b]$,使得
  \begin{equation*}
    f(\xi) = \frac{1}{n} [f(x_1) + \cdots + f(x_n)]
  \end{equation*}
\end{theorem}

\begin{proof}
  不妨设$f(x_1),\cdots,f(x_n)$中最大值为$f(x_n)$,最小值为$f(x_1)$,则
  \begin{equation*}
    f(x_1) \leq \frac{1}{n} [f(x_1) + \cdots + f(x_n)] \leq f(x_n)
  \end{equation*}
  若$f(x_1) = \frac{1}{n} \sum\limits_{i = 1}^n f(x_i)$或$f(x_n) = \frac{1}{n} \sum\limits_{i = 1}^n f(x_i)$,
  则取$\xi = x_1, x_n$。
  否则由连续函数介值性定理可证。
\end{proof}

\subsection{介值性定理的无连续推广}

有时没有连续的条件却能推出介值性定理的结论,常见的条件如单调。

~

\begin{exercise}[单调介值定理]
  核心:$f(x)$在$[0,1]$上单增,$f(0) > 0, f(1) < 1$,$k \in \mathbb{N}^+$,
  证明$\exists x_0 \in (0,1)$使得$f(x_0) = x_0^k$
\end{exercise}

\begin{proof}
  取$F(x) = f(x ) - x^k$,显然$F(0) = f(0) > 0, F(1) = f(1) - 1 < 0$,
  不停地二分取闭区间套,有$\lim \limits _{n \rightarrow \infty} |a_n - b_n| = 0$,
  $\lim \limits _{n \rightarrow \infty} a_n = \lim \limits _{n \rightarrow \infty} b_n = x_0$,
  $F(a_n) < 0, F(b_n) > 0 \Rightarrow f(a_n) > a_n^k, f(b_n) < b_n^k$,
  由于$f(x)$单增,$a_n^k < f(a_n) \leq f(x_0) \leq f(b_n) < b_n^k$,
  因此取极限可知$x_0^k \leq f(x_0) \leq x_0^k$得证。
\end{proof}

\begin{note}
  单调介值性定理不能将极限取进函数,但是用了两侧夹逼的方式。
\end{note}

~

\begin{exercise}
  证明:闭区间连续函数一定有界
\end{exercise}

\begin{proof}
  根据局部有界性以及有限开覆盖可证。
\end{proof}


\section{连续性的应用}


\begin{exercise}[几道经典题目]
  (1)$f(x)$在$x = 0$连续,$f(2x) = f(x)$,证明$f(x)$是常值函数

  (2)$f(x)$在$x = 0,1$连续,$f(x^2) = f(x)$,证明$f(x)$是常值函数

  (3)$f(x),g(x)$在$\mathbb{R}$上连续,且在每个有理点有$f(x) = g(x)$,证明$f(x)$恒等于$g(x)$

  (4)容易被吓到:$f(x) \in C[a,b]$,且对$\forall x \in [a,b]$都有$y \in [a,b]$使得$|f(y)| \leq \frac{1}{2}|f(x)|$,
  证明$f(x)$在$[a,b]$存在零点
\end{exercise}

\begin{proof}
  (1)$f(x) = f(\frac{x}{2}) = \cdots = f(\frac{x}{2^n}) = f(0)$,因此为常值

  (2)$x$为正直接$f(x) = f(x^{\frac{1}{2}}) = \cdots = f(x^{\frac{1}{2^n}}) = f(1)$,
  $x$为负平方一下转换为正,
  $x = 0$根据连续性可知等于附近,
  因此整体常数

  (3)根据有理点稠密性+连续性

  (4)假设无零点,根据闭区间连续函数有最小值,
  这与$|f(y)| \leq \frac{1}{2}|f(x)|$矛盾
\end{proof}


\begin{exercise}[几道经典最值问题]
  (1)重点:设$f(x)$在$(a,b)$连续,且$\lim \limits _{x \rightarrow a^+} f(x) = \lim \limits _{x \rightarrow b^-}f(x) = A$,
  这里$A$可以有限,也可以为$\pm \infty$,
  证明$f(x)$在$(a,b)$上有最大或最小值

  (2)重点:$f(x)$在$\mathbb{R}$上连续,$\lim \limits _{x \rightarrow +\infty}f(x) = \lim \limits _{x \rightarrow -\infty}f(x) = A$,
  这里$A$可以有限,也可以为$\pm \infty$,
  证明$f(x)$在$\mathbb{R}$上有最大或最小值

  (3)重点:$f(x)$在$[a,+\infty)$连续,$\lim \limits _{x \rightarrow +\infty}f(x) = A$,这里$A$为有限数,
  证明$f(x)$在$[a,+\infty)$存在最大值或最小值
\end{exercise}

\begin{proof}
  (1)对$A$进行分类讨论:

  (a)先考虑$A$为有限值。如果$f(x) \equiv A$,则显然。
  若$\exists x_0, f(x_0) > A$, 则$\exists \delta, \forall x \in (a, a+\delta) \cup (b-\delta,b)$满足$f(x) < f(x_0)$,
  由于$f(x)$在$[a+\delta,b-\delta]$上连续,
  因此$\exists \xi$使得$f(\xi)$在$[a+\delta,b-\delta]$最大,
  而$\forall x \in (a, a+\delta] \cup [b-\delta,b)$,自然也有$f(\xi) \geq f(x_0) \geq f(x)$,
  因此$f(\xi)$在$(a,b)$最大。
  若$f(x_0) < A$也同理有最小值。

  (b)若$A = +\infty$,则对$\forall x_0 \in (a,b)$,$f(x_0)$为有限数,
  $\exists \delta, \forall x \in (a, a+ \delta) \cup (b-\delta, b)$有$f(x) > f(x_0)$,
  由于$f(x)$在$[a+\delta,b-\delta]$连续,因此有最小值,且最小值比$f(x_0)$还小,因此在$(a,b)$上取最小值。

  (2)和(1)同理,只是把$\exists \delta$换成$\exists N$

  (3)若$f(x) \equiv A$,则显然。
  若$\exists x_0, f(x_0) > A$,则$\exists M, \forall x > M$有$f(x_0) \geq f(x)$,
  由于在$[a,M]$连续,有最大值$f(\xi)$,
  且$\forall x >M, f(\xi) > f(x_0) > f(x)$,因此有最大值。
  最小值同理
\end{proof}



\section{一致连续}

\subsection{一致连续的概念}

\begin{definition}[一致连续]
  $I$是任意区间(任意开闭,有界无界),
  $f: I \rightarrow \mathbb{R}$,
  $\forall \epsilon , \exists \delta$对$\forall x_1,x_2 \in I$满足$|x_1 - x_2| < \delta$时有$|f(x_1) - f(x_2)| < \epsilon$,则称$f$在$I$一致连续
\end{definition}

\begin{theorem}[常用充要条件]
  $f(x)$在区间$I$上有定义,
  则$f(x)$在$I$一致连续当且仅当任意数列$x_n^{\prime},x_n^{\prime\prime}$,
  若$\lim \limits _{n \rightarrow \infty} x_n^{\prime} - x_n^{\prime\prime} = 0$,
  则$\lim \limits _{n \rightarrow \infty} [f(x_n^{\prime}) - f(x_n^{\prime\prime})] = 0$
\end{theorem}

\begin{proof}
  (1)左推右:显然

  (2)右推左:反设不一致连续,$\exists \epsilon_0, \forall \delta, \exists x^{\prime}, x^{\prime\prime}$,
  虽然$|x^{\prime} - x^{\prime\prime}| < \delta$,但是
  $|f(x^{\prime}) - f(x^{\prime\prime})| \geq \epsilon_0$。
  取$\delta = 1,\frac{1}{2},\cdots,\frac{1}{n},\cdots$,
  $\exists x_n^{\prime},x_n^{\prime\prime}$,$|x_n^{\prime} - x_n^{\prime\prime}| < \frac{1}{n}, |f(x_n^{\prime}) - f(x_n^{\prime\prime})| \geq \epsilon_0$,
  这与条件矛盾。
\end{proof}

\begin{note}
  证明不一致连续绝大多数情况都用上述定理。
\end{note}

~

\begin{exercise}
  连续但不一致连续的例子:
  (1)$x^2, (0,+\infty)$
  (2)$\sin x^2, [0,+\infty)$
  (3)$\frac{x+2}{x+1}\sin \frac{1}{x}, (0,1)$
  (4)$\frac{1}{x}, x \in (0,1)$
\end{exercise}

\begin{proof}
  (1)$x_n = \sqrt{n}, y_n = \sqrt{n+1}, \lim \limits _{n \rightarrow \infty} (x_n - y_n) = 0, \lim \limits _{n \rightarrow \infty} [f(x_n) - f(y_n)]  = -1$

  (2)取$x_n = \sqrt{2n\pi}, y_n = \sqrt{2n\pi + \frac{\pi}{2}}$

  (3)取$x_n = \frac{1}{2n\pi}, y_n = \frac{1}{2n\pi + \frac{\pi}{2}}$

  (4)取$x_n = \frac{1}{n}, y_n = \frac{1}{n+1}$
\end{proof}

~

\begin{exercise}[一致连续的证明]
  (1)证明$f(x) = \cos \sqrt{x}$在$[0,+\infty)$一致连续

  (2)证明$f(x) = \sqrt{x}$在$[0,+\infty)$一致连续

  (3)证明$f(x) = \sin x$在$[0,+\infty)$一致连续
\end{exercise}

\begin{proof}
  (1)方法1:不妨设$x_1 > x_2$,则
  \begin{align*}
    |f(x_1) - f(x_2)| &= |\cos \sqrt{x_1} - \cos \sqrt{x_2}| = 2 \left| \sin \frac{\sqrt{x_1} + \sqrt{x_2}}{2} \sin \frac{\sqrt{x_1} - \sqrt{x_2}}{2} \right|\\
   & \leq 2 \left| \sin \frac{\sqrt{x_1} - \sqrt{x_2}}{2} \right| \leq |\sqrt{x_1} - \sqrt{x_2}| = \frac{|x_1 - x_2|}{\sqrt{x_1} + \sqrt{x_2}}\\
    & = \frac{x_1 - x_2}{\sqrt{(x_1 - x_2) + x_2} + \sqrt{x_2}} \leq \frac{x_1 - x_2}{\sqrt{x_1 - x_2}} = \sqrt{x_1 - x_2}
  \end{align*}
  当$\delta = \epsilon^2$时,$|x_1 - x_2| < \delta$有$|f(x_1) - f(x_2)| < \epsilon$,
  因此一致连续

  方法2:$(\cos \sqrt{x})^{\prime} = \sin \sqrt{x} \frac{1}{2 \sqrt{x}}$在$[1,+\infty)$有界,
  而$[0,1]$用Cantor定理,
  因此Lip连续推出一致连续

  (2)分段+Lip连续
\end{proof}


\subsection{Cantor定理及其推广}

\begin{theorem}[Cantor定理]
  若$f(x)$在闭区间$[a,b]$连续,则其在$[a,b]$上一致连续。
\end{theorem}

\begin{proof}
  反设在$[a,b]$不一致连续,
  根据一致连续充要条件,
  $\exists \epsilon_0, \exists x_n,y_n$满足
  $\lim \limits _{n \rightarrow \infty} (x_n - y_n) = 0$,
  但$|f(x_n) - f(y_n)| \geq \epsilon_0$。
  由于$x_n$有界,根据致密性定理可知$x_n$有收敛子列,
  设$\lim \limits _{k \rightarrow \infty}x_{n_k} = x_0$,
  从而$\lim \limits _{k \rightarrow \infty}y_{n_k} = \lim \limits _{k \rightarrow \infty}[(y_{n_k} - x_{n_k}) + x_{n_k}] = x_0$,
  因此推出$\epsilon_0 \leq \lim \limits _{k \rightarrow \infty}|f(x_{n_k}) - f(y_{n_k})| = |f(x_0) -f(x_0)| = 0$(这一步用了连续条件),
  矛盾。
\end{proof}

\begin{note}
  Cantor定理的其他证明方式:
  \begin{itemize}
  \item 有限覆盖定理:$\forall \epsilon, \exists \delta_{x_0}, \forall x \in B(x_0,\delta_{x_0})$,有$|f(x) - f(x_0)| < \epsilon$,取遍闭区间,有有限覆盖,
    上述$\epsilon$不变,当取$\delta = \min \{\delta_x\}$时,即有$\forall a,b $满足$|a-b| < \delta$都有$|f(a) - f(b)| < \epsilon$
  \item 闭区间套:$f(x)$在$[a,b]$不一致连续,则一定在$[a,\frac{a+b}{2}],[\frac{a+b}{2},b]$中有一个区间中不一致连续,
    不断取区间套,最后套住$\xi$,由于$f$在$\xi$连续,$\forall \epsilon, \exists \delta, \forall x, |x - \xi| < \delta$有$|f(\xi) - f(x)| < \frac{\epsilon}{2}$,
    同理取$y$,得到$|f(x )- f(y)| < \epsilon$,
    而存在$a_n,b_n$满足$[a_n,b_n] \subseteq [\xi - \delta, \xi + \delta]$,
    这与不一致连续矛盾
  \item 单调有界定理:和闭区间套一种构造方式
  \end{itemize}
\end{note}



\begin{theorem}[开区间一致连续充要条件]
  $f(x)$在$(a,b)$连续,则$f(x)$在$(a,b)$一致连续的充要条件是$\lim \limits _{x \rightarrow a^+} f(x), \lim \limits _{x \rightarrow b^-}f(x)$存在(有限)
\end{theorem}

\begin{proof}
  (1)左推右:根据$f(x)$在$(a,b)$一致连续,
  $\forall \epsilon, \exists \delta, \forall x^{\prime},x^{\prime\prime}, |x^{\prime} - x^{\prime\prime}| < \delta$,有$|f(x^{\prime}) - f(x^{\prime\prime})| < \epsilon$。
  取$x^{\prime},x^{\prime\prime} \in (a,a + \delta)$,显然$|f(x^{\prime}) - f(x^{\prime\prime})| < \epsilon$,
  即右极限的Cauchy准则,另一侧同理。

  (2)右推左:补充定义两个端点,根据Cantor定理可知。
\end{proof}

~

\begin{exercise}[开区间一致连续应用]
  (1)(东北师大)设$f(x)$在$(a,b)$上一致连续,
  证明$f(x)$在$(a,b)$上有界
\end{exercise}

\begin{proof}
  (1)由于一致连续,因此$f(x)$在$a,b$的分别存在右、左极限,
  补充定义为$F(x)$,
  $F(x)$在$[a,b]$上连续有界,
  且$f(x)$和$F(x)$在$(a,b)$相等,
  因此$f(x)$在$(a,b)$有界。
\end{proof}

~

\begin{theorem}[无穷区间一致连续]
  $f(x)$在$[a,+\infty)$连续,
  $\lim \limits _{x \rightarrow +\infty}f(x) = A$(有限),
  则$f(x)$在$[a,+\infty)$一致连续
\end{theorem}

\begin{proof}
  由于$\lim \limits _{x \rightarrow +\infty}f(x) = A$,根据Cauchy收敛准则,
  $\forall \epsilon, \exists M, \forall x^{\prime},x^{\prime\prime} > M$有$|f(x^{\prime}) - f(x^{\prime\prime})| < \epsilon$。
  根据Cantor定理,$f(x)$在$[a,M+1]$一致连续,
  对$\forall \epsilon, \exists \delta \in (0,1)$,只要$|x^{\prime} - x^{\prime\prime}| < \delta$时,
  两点要么同时落在$[a,M+1]$,要么同时落在$[M,+\infty)$,根据上面的命题可知一致连续。
\end{proof}

\begin{note}
  $[a,+\infty)$一致连续并不代表无穷处$f(x)$极限一定存在,只是一个充分条件,例如$f(x) = x$极限不存在,但一致连续。
\end{note}

~

\begin{exercise}[开区间一致连续]
  证明:(1)$x\sin \frac{1}{x}$在$(0,1)$一致连续(2)$\sin \frac{1}{x}$在$(0,1)$不一致连续
\end{exercise}

\begin{proof}
  (1)因为$\lim \limits _{x \rightarrow 0}x \sin \frac{1}{x} = 1$,因此一致连续

  (2)因为$\lim \limits _{x \rightarrow 0}\sin \frac{1}{x}$不存在,故不一致连续。
\end{proof}

~


\subsection{Lip连续与一致连续}

\begin{theorem}[Lip连续与一致连续]
  $f(x)$在区间$I$上满足Lip条件,
  则$f(x)$在$I$上一致连续
\end{theorem}

\begin{proof}
  Lip条件即$\forall x^{\prime},x^{\prime\prime}$有$|f(x^{\prime}) - f(x^{\prime\prime})| \leq L |x^{\prime}  - x^{\prime\prime}|$,
  取$\forall \epsilon, \delta = \frac{\epsilon}{L}$即可。
\end{proof}

\begin{corollary}[导数有界一定一致连续]
  若$f^{\prime}(x)$在$I$上有界,则根据Lagrange中值定理可知Lip连续,从而一致连续
\end{corollary}

\begin{corollary}[导数在无穷处趋于无穷则不一致连续]
  若$\lim \limits _{x \rightarrow +\infty}f^{\prime}(x) = \infty$,
  则$f(x)$在$[a,+\infty)$不一致连续
\end{corollary}

\begin{proof}
  根据$\lim \limits _{x \rightarrow +\infty}|f^{\prime}(x)| = \infty$可知$|x^{\prime} - x^{\prime\prime}|$控制不住导数的增长。
\end{proof}

\begin{note}
  注意上述两个推论都是充分条件。
  前者例如$\sqrt{x}$在$[0,1]$一致连续,其导数$\frac{1}{2 \sqrt{x}}$却不有界。
  后者例如$x^2$在无穷处导数趋于无穷,则一定不一致连续。
\end{note}

~

\begin{exercise}[基础练习]
  判断一致连续性:
  (1)$\sin x, x \in \mathbb{R}$
  (2)$x^k, x \in (1,+\infty)$
  (3)$\cos \sqrt{x}, x \in (1, +\infty)$
  (4)$x\ln x, x \in (0,+\infty)$
\end{exercise}

\begin{solution}
  (1)一致连续,因为导数有界

  (2)若$k \leq 0$,则$\lim \limits _{x \rightarrow \infty}f(x)$存在,补充定义$x = 1$,根据Cantor定理的无穷推广可知一致连续。
  若$0 < k \leq 1$,则导数有界,因此一致连续。
  若$k > 1$,因为导数在无穷处趋于无穷,则不一致连续

  (3)导数有界,一致连续

  (4)导数在无穷处趋于无穷,则不一致连续。
\end{solution}

\subsection{区间合并问题}

\begin{theorem}[一致连续区间合并定理]
  设区间$I_1$右端点为$c \in I_1$(闭),$I_2$左端点为$c \in I_2$(闭),
  $f(x)$在$I_1,I_2$均一致连续,则其在$I_1 \cup I_2$也一致连续
\end{theorem}

\begin{proof}
  由于$f(x)$在$I_1,I_2$,对$\forall \epsilon$,存在公共$\delta_1$,
  当$x^{\prime},x^{\prime\prime}$同时落于$I_1$或$I_2$时,
  若$|x^{\prime} - x^{\prime\prime}| < \delta_1$,则$|f(x^{\prime}) - f(x^{\prime\prime})| < \epsilon$。
  若$x^{\prime},x^{\prime\prime}$分别落于$I_1,I_2$中,
  则$\exists \delta_2, |x^{\prime} - c| < \delta_2$时,
  $|f(x^{\prime}) - f(c)| < \frac{\epsilon}{2}$且$|x^{\prime\prime} - c| < \delta_2$时$|f(x^{\prime\prime}) - f(c)| < \frac{\epsilon}{2}$,
  取$\delta = \min \{\delta_1,\delta_2\}$,
  得出$\forall x^{\prime},x^{\prime\prime} \in I_1 \cup I_2$,当$|x^{\prime} - x^{\prime\prime}| < \delta$时
  $|f(x^{\prime}) - f(x^{\prime\prime})| < \epsilon$
\end{proof}

\begin{note}
  如果边界点为开区间点,则可以补上定义。
\end{note}

~

\begin{exercise}[区间合并练习]
  (1)考虑$f(x) = \frac{|\sin x|}{x}$在$(-1,0) \cup (0,1)$上是否一致连续
\end{exercise}

\begin{proof}
  (1)由于$\lim \limits _{x \rightarrow 0^+}f(x) = 1, \lim \limits _{x \rightarrow 0^-}f(x) = -1$,
  $\lim \limits _{x \rightarrow 1^-}f(x), \lim \limits _{x \rightarrow -1^+}f(x)$显然存在,
  因此$f(x)$在$(-1,0),(0,1)$显然一致连续。
  但是由于$0$处左右极限不相同,
  取$x_n$单减趋于$0$,$y_n$单增趋于$0$,则$|x_n - y_n| \rightarrow 0$,而$|f(x_n) - f(y_n)| \rightarrow 2$,
  因此不一致连续。
\end{proof}

\subsection{四则运算、复合的一致连续性}

\begin{theorem}[四则运算的一致连续性]
  设$f(x),g(x)$在区间$I$上一致连续,则
  \begin{itemize}
  \item 数乘、加减:$\forall k \in \mathbb{R}$,$kf(x), f(x) \pm g(x)$在$I$一致连续 
  \item 乘法:$I$为有限区间时,$f(x)g(x)$在$I$一致连续。$I$为无穷区间时,$f(x)g(x)$在$I$不一定一致连续
  \item 除法:$f(x) \neq 0$,$\frac{1}{f(x)}, \frac{g(x)}{f(x)}$不一定一致连续
  \end{itemize}
\end{theorem}

\begin{proof}
  (1)直接三角不等式放缩

  (2)$|f(x^{\prime})g(x^{\prime}) - f(x^{\prime\prime})g(x^{\prime\prime})| \leq |f(x^{\prime})g^{\prime}(x) - f(x^{\prime}) g(x^{\prime\prime})| + |f(x^{\prime})g(x^{\prime\prime}) - f(x^{\prime\prime})g(x^{\prime\prime})| \leq |f(x^{\prime})||g(x^{\prime}) - g(x^{\prime\prime})| + |g(x^{\prime\prime})||f(x^{\prime}) - f(x^{\prime\prime})| < 2M\epsilon$,
  因此若$M$有限显然成立,$M$不有限时不一定。
\end{proof}

~

\begin{example}[四则运算反例]
  (1)乘法:$x$在$(1,+\infty)$一致连续,但是$x^2$在$(1,+\infty)$不一致连续
  (2)倒数:$x$在$(0,1)$一致连续,但是$\frac{1}{x}$在$(0,1)$不一致连续
\end{example}

~

\begin{exercise}[四则运算练习]
  $f(x)$在$[a,+\infty)$连续,
  $g(x)$在$[a,+\infty)$一致连续,
  $\lim \limits _{x \rightarrow \infty}[f(x) - g(x)] = 0$,
  证明:$f(x)$在$[a,+\infty)$上一致连续
\end{exercise}

\begin{proof}
  由于$f(x) - g(x)$在无穷远处极限存在,其一致连续。
  $g(x)$一致连续,
  根据四则运算可知$f(x) = [f(x) - g(x)] + g(x)$一致连续。
\end{proof}

\begin{note}
  上述练习本身也是一个重要的结论。
\end{note}

~

\begin{theorem}[复合的一致连续性]
  $f(x)$在$I_1$一致连续,$g(x)$在$I_2$一致连续,
  $f(x)$值域在$I_2$中,则$g(f(x))$在$I_1$一致连续
\end{theorem}

\begin{proof}
  根据定义套即可。
  外层取$\epsilon, \delta$,内层取$\delta, \delta^{\prime}$。
\end{proof}

~

\begin{exercise}[复合一致连续性]
  设$f(x)$在$[0,+\infty)$满足Lip条件,
  $\alpha \in (0,1)$,证明:$f(x^{\alpha})$在$[0,+\infty)$一致连续
\end{exercise}

\begin{proof}
  $\alpha \in (0,1)$时,$x^{\alpha}$在$[0,+\infty)$一致连续,
  而$f$满足Lip条件也一致连续,
  根据复合一致连续即可。
\end{proof}

% \subsection{周期函数的一致连续性}

% \begin{theorem}[周期函数的一致连续性]
%   $f(x)$时$\mathbb{R}$上以$T$为周期的连续函数,
%   则$f(x)$在$\mathbb{R}$上一致连续。
% \end{theorem}

% \begin{proof}
%   由于在$[0,2T]$一致连续,对$\forall \epsilon, \exists \delta \in (0,T), x^{\prime},x^{\prime\prime} \in [0,2T]$满足$|x^{\prime} - x^{\prime\prime}| < \delta$时,
%   $|f(x^{\prime}) - f(x^{\prime\prime})| < \epsilon$。
%   对$\forall x^{\prime} , x^{\prime\prime} \in \mathbb{R}$,
%   有$x^{\prime} - nT \in [0,T], x^{\prime\prime} - nT \in [0,2T]$,
%   因此转换为$[0,2T]$的问题。
% \end{proof}



\section{一致连续的经典问题}


\begin{exercise}
  设$f(x)$在$\mathbb{R}$上一致连续,
  证明$\exists a,b > 0$使得$|f(x)| \leq a|x| + b$
\end{exercise}

\begin{proof}
  由于$f(x)$一致连续,对$\epsilon = 1, \exists \delta$,当$|x^{\prime} - x^{\prime\prime}| < \delta$时
  $|f(x^{\prime}) - f(x^{\prime\prime})| \leq 1$,
  不妨设
  \begin{equation*}
    |f(x)| \leq M, x \in [-\delta, \delta]
  \end{equation*}
  对$\forall x > \delta, \exists n_x$使得$x - n_x\delta = x_0  \in [0,\delta)$,
  此时$n_x = \frac{x - x_0}{\delta} \leq \frac{|x|}{\delta}$,
  因此
  \begin{equation*}
    |f(x)| \leq \sum\limits_{k = 0}^{n-1}|f(x-k\delta) - f(x - (k+1)\delta)| + |f(x - n_x\delta)| \leq n_x + M \leq \frac{|x|}{\delta} + M
  \end{equation*}
  同理$\forall x < - \delta$也有$|f(x)| < \frac{|x|}{\delta} + M$,
  综上,取$a = \frac{1}{\delta}, b = M$即可。
\end{proof}

% ~

% \begin{exercise}[$\frac{f(x)}{x}$的性质]
%   (1)设$f(x)$在$[a,+\infty)$上一致连续,$a > 0$,证明$\frac{f(x)}{x}$在$[a,+\infty)$有界

%   (2)设$f(x)$在$[a,+\infty)$上Lip连续,其中$a > 0$,证明$\frac{f(x)}{x}$在$[a,+\infty)$一致连续
% \end{exercise}









