
\chapter{定积分}

\section{定积分理论}

\subsection{可积性}

\begin{definition}[Riemann积分]
  $f(x)$是$[a,b]$上的函数,将$[a,b]$划分为$n$等份$T$,
  若$\lim \limits _{||T|| \rightarrow 0} \sum\limits_{i=1}^n f(\xi_i) \Delta x_i = J$,
  这里$J$为常数,则称$f(x)$在$[a,b]$可积,
  且记$\int_a^bf(x)\mathrm{d}x = J$
\end{definition}

\begin{definition}[Darboux上下和]
  在每个区间上取$M_i = \sup \limits _{x \in \Delta _i} f(x), m_i = \inf \limits_{x \in \Delta _i}f(x)$,
  记振幅$\omega_i = M_i - m_i$,
  定义Darboux上下和为
  \begin{equation*}
    S(T) = \sum\limits_{i = 1}^n M_i\Delta x_i, s(T) = \sum\limits_{i = 1}^n m_i\Delta x_i
  \end{equation*}
  定义$f(x)$在$[a,b]$的上下积分为:
  \begin{equation*}
    S = \inf \limits_T S(T), s = \sup \limits_T s(T)
  \end{equation*}
\end{definition}

\begin{theorem}[可积函数的性质]
  若$f(x)$在$[a,b]$上可积,则:
  \begin{itemize}
  \item 有界性:$f(x)$在$[a,b]$上有界(反之不正确,有界不一定可积,例如Dirichlet函数)
  \end{itemize}
\end{theorem}

\begin{proof}
  反设$f$无界,则存在某个小区间$\Delta_k$,使得$\Delta_k$上$f$无界,
  假设其他区间都有界,
  此时Riemann和
  \begin{equation*}
    |\sum\limits_{i = 1}^n f(\xi_i) \Delta x_i| \geq |f(\xi_k) \Delta x_k| - |\sum\limits_{i \neq k} f(\xi_i)\Delta x_i|
  \end{equation*}
  对$\forall M > 0$,无论如何分割,Riemann和都能大于$M$,
  因此无界不可积。
\end{proof}

\begin{theorem}[可积性充要条件]
  $f(x)$在$[a,b]$可积当且仅当
  \begin{itemize}
  \item 上下积分角度:$f(x)$在$[a,b]$的上积分$s$等于下积分$S$
  \item 上下和角度(最常用!):$\forall \epsilon$,存在分割$T$使得$S(T) - s(T) = \sum\limits_T \omega_i \Delta x_i< \epsilon$
  \item 振幅角度:对$\forall \epsilon, \eta$,总存在分割$T$,使得振幅$\omega_{k^{\prime}} \geq \epsilon$的那些小区间总长$\sum\limits \omega_{k^{\prime}} < \eta$
  \item 实变角度:$f(x)$在$[a,b]$几乎处处连续
  \end{itemize}
\end{theorem}

~

\begin{exercise}[Dirichlet函数的不可积性]
  证明Dirichlet函数在$[a,b]$不可积
\end{exercise}

\begin{proof}
  由于任意分割$T$,振幅$\omega = 1$,因此$S(T) - s(T) = \sum\limits_{i = 1}^n \Delta x_i = b - a$,因此不可积。
\end{proof}

~

\begin{theorem}[Riemann可积函数类]
  $f(x)$定义于$[a,b]$,
  常见的有三个Riemann可积函数类:
  \begin{itemize}
  \item $[a,b]$连续函数
  \item 在$[a,b]$只有有限个间断点的有界函数
  \item $[a,b]$上单调函数
  \end{itemize}
\end{theorem}

\begin{proof}
  (1)根据一致连续性,每个$\omega_i$都可以足够小,因此$\sum \omega_i \Delta x_i$足够小

  (2)将间断点处的$\Delta x_i$设置得足够小,控制住间断点差距

  (3)$\omega_i = f(x_i) - f(x_{i-1})$,只需要$\omega_i$取得足够小
\end{proof}

~

\begin{exercise}[可积性判断]
  (1)证明Riemann函数$R(x)$在$[0,1]$可积,且$\int_0^1 R(x)\mathrm{d}x = 0$

  (2)证明下面的$f(x)$在$[0,1]$上可积:
  \begin{equation*}
    f(x) =
    \begin{cases}
      0, & x = 0\\
      \frac{1}{x} - \left[ \frac{1}{x} \right], & x \in (0,1]
    \end{cases}
  \end{equation*}
\end{exercise}

\begin{proof}
  (1)只需证$\sum \omega_i \Delta x_i < \epsilon$,
  给定$\forall \epsilon$,满足$\frac{1}{q} > \frac{\epsilon}{2}$的只有有限个,
  且它们的函数值均小于等于$\frac{1}{2}$。
  把这有限个全部列出$x_1, \cdots, x_k$,
  这些点每个最多落入两个小区间中,
  取$||T|| < \frac{\epsilon}{2k}$即可,则该部分$\sum \omega_i \Delta x_i < \frac{\epsilon}{2}$。
  另一侧直接$\sum \omega_i \Delta x_i < \frac{\epsilon}{2} \sum \Delta x_i \leq \frac{\epsilon}{2}$就行。

  (2)$f(x)$振幅小于等于$1$,不连续点为$0,\frac{1}{2},\frac{1}{3},\cdots,\frac{1}{n},\cdots$。
  对$\forall \epsilon$,$\exists \delta_1$,当$|\Delta x_i| < \delta_1$时,
  $(\epsilon, 1)$分割满足$\sum \omega_i \Delta x_i < \epsilon$。
  而$(0,\epsilon)$的任意分割也满足$\sum \omega_i \Delta x_i < \epsilon$,
  因此取$\delta = \min \{\delta_1,\epsilon\}$时,整体的分割满足:
  \begin{equation*}
    \sum \omega_i \Delta x_i < \epsilon
  \end{equation*}
\end{proof}

\begin{exercise}[可积性相关证明]
  (1)$f(x)$在$[a,b]$有定义,$\forall \epsilon$,存在$[a,b]$上可积函数$g(x)$,使得$\forall x \in [a,b], |f(x) - g(x)| < \epsilon$,证明$f(x)$在$[a,b]$可积
\end{exercise}

\begin{proof}
  (1)由于$g(x)$可以,存在分割$T$使得$\sum\limits_T \omega_i^g \Delta x_i < \epsilon$,
  而$\omega_i^f \leq \omega_i^{f-g} + \omega_i^g$,
  以及$\omega_i^{f-g} \leq 2\epsilon$(因为$g$本身可能$\epsilon$震荡,$|f(x)-g(x)|$又可能$\epsilon$震荡),
  得到
  \begin{equation*}
    \sum\limits_T \omega_i^f \Delta x_i \leq \sum\limits_T \omega_i^{f-g} \Delta x_i + \sum\limits_T \omega_i^g \Delta x_i < 2 \epsilon(b - a) + \epsilon
  \end{equation*}
\end{proof}

\subsection{Newton-Leibniz公式}

\begin{theorem}[Newton-Leibniz公式]
  $f(x)$在$[a,b]$可积,$F(x)$在$[a,b]$上连续,
  且除有限个点外,$F^{\prime}(x) = f(x)$,
  则
  \begin{equation*}
    \int_a^b f(x)\mathrm{d} x = F(b) - F(a)
  \end{equation*}
\end{theorem}

\begin{note}
  Newton-Leibniz公式的意义在于将定积分的计算转换为了原函数的计算,
  不过可积与存在原函数也没有直接关系,具体见下面题目。
\end{note}

~

\begin{exercise}[可积与原函数]
  举例说明可积未必有原函数,有原函数未必可积
\end{exercise}

\begin{solution}
  (1)例如$[-1,1]$上的$\text{sgn}(x)$,
  其可积但无原函数

  (2)例如原函数如下,其在$[0,1]$可导,但是$F^{\prime}(x)$无界
  \begin{equation*}
    F(x) = 
    \begin{cases}
      0, & x = 0\\
      x^{\frac{3}{2}} \sin \frac{1}{x}, & x \in (0,1]
    \end{cases}
  \end{equation*}
\end{solution}

~

\begin{exercise}[使用Nowton-Leibniz公式]
  (1)已知$f(x) =
  \begin{cases}
    2022, &  x = 0\\
    2x \sin \frac{1}{x} - \cos \frac{1}{x}, & x \in (0,1]
  \end{cases}
  $,证明$f(x)$在$[0,1]$可积,求$\int_0^1 f(x)\mathrm{d} x$
\end{exercise}

\begin{solution}
  (1)$f(x)$在$0$处为跳跃间断点,只有一处,而$[0,1]$上$f(x)$有界,故可积。
  构造$F(x)$:
  \begin{equation*}
    F(x) =
    \begin{cases}
      0, & x = 0\\
      x^2 \sin \frac{1}{x}, & x \in (0,1]
    \end{cases}
  \end{equation*}
  $F(x)$除了$0$处外满足$F^{\prime}(x) = f(x)$,因此$\int_0^1 f(x)\mathrm{d} x = F(1) - F(0) = \sin 1$
\end{solution}

~

\begin{theorem}[微积分基本定理]
  $f(x)$为$[a,b]$可积函数,定义$F(x) = \int_a^x f(t)\mathrm{d} t, x \in [a,b]$,则
  \begin{itemize}
  \item $F(x)$是$[a,b]$上的连续函数
  \item 若$x_0 \in [a,b]$为$f(x)$连续点,则$F(x)$在$x_0$可导,且$F^{\prime}(x_0) = f(x_0)$
  \item 若$f(x)$为$[a,b]$上连续函数,则$F(x)$为$[a,b]$上连续可微函数,且$F^{\prime}(x) = f(x)$
  \end{itemize}
\end{theorem}

~

\begin{exercise}[使用微积分基本定理]
  (1)重点:$f(x)$在$[A,B]$上可积,$a,b \in (A,B)$是$f(x)$的两个连续点,证明
  \begin{equation*}
    \lim \limits _{h \rightarrow 0} \int_a^b \frac{f(x+h) - f(x)}{h} \mathrm{d} x = f(b) - f(a)
  \end{equation*}
\end{exercise}

\begin{proof}
  (1)对左侧做如下化简:
  \begin{align*}
    \int_a^b \frac{f(x+h) - f(x)}{h}\mathrm{d} x &= \frac{1}{h} \left( \int_a^b f(x+h)\mathrm{d} x - \int_a^b f(x)\mathrm{d}x \right)\\
    &= \frac{1}{h} \left( \int_{a+h}^{b+h} f(x)\mathrm{d} x - \int_a^b f(x)\mathrm{d} x \right)\\
    &= \frac{1}{h} \left( \int _b^{b+h} f(x)\mathrm{d} x - \int_a^{a+h} f(x)\mathrm{d} x \right)
  \end{align*}
  令$F(x) = \int _A^x f(t)\mathrm{d} t$,
  则显然$F^{\prime}(b) = \lim \limits _{h \rightarrow 0^+} \frac{F(b+h) - F(b)}{h} = f(b)$,
  $F^{\prime}(a) = f(a)$同理。
\end{proof}

\subsection{四则运算与复合}

\begin{theorem}[四则运算的可积性]
  若$f(x),g(x)$在$[a,b]$可积,
  则$kf(x), f(x) \pm g(x), f(x)g(x)$在$[a,b]$可积,
  但$g(x) \neq 0$时$\frac{f(x)}{g(x)}$也不一定可积
\end{theorem}

\begin{proof}
  (1)数乘、加减显然

  (2)乘法:根据可积可知有界,设$|f(x)| \leq M, |g(x)| \leq N$,
  $\sum\limits_T \omega_i^{fg} \Delta x_i < M\sum\limits_T \omega_i^g \Delta x_i + N \sum\limits_T \omega_i^f \Delta x_i$即可

  (3)除法:例如$f(x) = 1,g(x) =
  \begin{cases}
    x, & x \in (0,1]\\
    1, & x = 0
  \end{cases}
  $,$\frac{1}{g(x)}$因为无界而不可积
\end{proof}

~

\begin{exercise}[除法增加条件可积]
  设$f(x)$在$[a,b]$上函数,证明

  (1)若$f(x)$可积,且$|f(x)| \geq m > 0$,则$\frac{1}{f(x)}$在$[a,b]$可积

  (2)若$f(x)$在$[a,b]$连续且恒不为零,则$\frac{1}{f(x)}$在$[a,b]$可积
\end{exercise}

\begin{proof}
  (1)考虑振幅放缩:
  \begin{equation*}
    \left| \frac{1}{f(x^{\prime}) } - \frac{1}{f(x^{\prime\prime})} \right| = \left| \frac{f(x^{\prime\prime}) - f(x^{\prime})}{f(x^{\prime})f(x^{\prime\prime})} \right| \leq \frac{1}{m^2}\omega_i^f
  \end{equation*}

  (2)由于连续,能找到最小值,可转换为(1)
\end{proof}

~

\begin{theorem}[复合(较难,重点!)]
  $f(x)$在$[a,b]$连续,$\varphi(t)$在$[\alpha,\beta]$可积,
  且$a \leq \varphi(t) \leq b, t \in [\alpha, \beta]$,
  则$f(\varphi(t))$在$[\alpha,\beta]$可积
\end{theorem}

\begin{proof}
  $f(x)$在$[a,b]$连续,因此有界且一致连续。
  设$M = \max \limits _{x \in [a,b]}f(x)$,
  $\forall \epsilon, \exists \delta, \forall x^{\prime},x^{\prime\prime} \in [a,b]$,
  若$|x^{\prime} - x^{\prime\prime}| < \delta$就有$|f(x^{\prime} ) - f(x^{\prime\prime})| < \epsilon$。
  由于$\varphi$可积,存在分割$T$,使得$\sum\limits_{\omega_i^{\varphi} \geq \delta} \Delta t_i < \epsilon$,
  因此
  \begin{align*}
    \sum\limits_T \omega_i^{f \circ \varphi} \Delta t_i &= \sum\limits_{\omega_i^{\varphi} \geq \delta} \omega_i^{f \circ \varphi} \Delta t_i + \sum\limits_{\omega_i^{\varphi} < \delta} \omega_i^{f \circ \varphi} \Delta t_i\\
    &\leq 2M \sum\limits_{\omega_i^{\varphi} \geq \delta} \Delta t_i + \epsilon \sum\limits_{\omega_i^{\varphi} < \delta} \Delta t_i\\
    &< 2M \epsilon + (\beta - \alpha)\epsilon = \epsilon^{\prime}
  \end{align*}
\end{proof}

\subsection{定积分的保号性}

\begin{theorem}[定积分保号性]
  $f(x)$在$[a,b]$连续,
  $f(x) \geq 0$,则$\int_a^b f(x)\mathrm{d} x \geq 0$,
  且$\int_a^b f(x)\mathrm{d} x = 0$当且仅当$f(x) \equiv 0$
\end{theorem}

~

\begin{exercise}[等于0的推广]
  (1)重点:$f(x)$在$[a,b]$连续,且对任意满足$\int_a^b g(x)\mathrm{d} x = 0$的连续函数$g(x)$有
  $\int_a^b f(x)g(x)\mathrm{d} x = 0$,证明:$f(x)$为常值函数

  (2)重点(ZJU2020):
  $f(x)$在$[a,b]$可导,$f(a) = 0$,
  对$\forall x \in [a,b]$有$|f(x)| \geq |f^{\prime}(x)|$,证明:$f(x) \equiv 0$
\end{exercise}

\begin{proof}
  (1)取$g(x) = f(x) - \frac{1}{b-a}\int_a^b f(t)\mathrm{d} t$,
  显然$\int_a^b g(x)\mathrm{d} x = 0$满足条件,
  此时
  \begin{equation*}
    \int_a^b g^2(x)\mathrm{d}x = \int _a^b g(x) \left( f(x) - \frac{1}{b-a} \int_a^b f(x)\mathrm{d} x \right) = 0
  \end{equation*}

  (2)可证明$\int_a^x f(t)\mathrm{d} t = 0, \forall x \in [a,b]$,则考虑$g(x) = \int_a^x |f(t)|\mathrm{d} t$,有
  \begin{equation*}
    g(x) \geq \int_a^x|f^{\prime}(t)|\mathrm{d} t \geq \left| \int_a^x f^{\prime}(t)\mathrm{d} t \right| = |f(x) | = g^{\prime}(x)
  \end{equation*}
  因此得到$(e^{-x}g(x))^{\prime} = e^{-x}(g^{\prime}(x) - g(x)) \leq 0$,即$e^{-x}g(x)$单调减,
  而$e^{-a}g(a) = 0$,且$e^{-x}g(x) \geq 0$,综上得到$g(x) \equiv 0$
\end{proof}

~

\begin{exercise}[零点问题]
  (1)$f(x)$为$[0,\pi]$上连续函数,且$\int_0^{\pi} f(x) \sin x\mathrm{d} x = \int_0^{\pi} f(x) \cos x \mathrm{d} x = 0$,
  证明$f(x)$在$(0,\pi)$上至少有两个零点。

  (2)$f(x)$为$[a,b]$上连续函数,
  $\int_a^b f(x)\mathrm{d} x = \int_a^b xf(x)\mathrm{d} x = \cdots = \int_a^b x^n f(x)\mathrm{d}x = 0$,
  证明$f(x)$在$(a,b)$上至少有$n+1$个零点
\end{exercise}

\begin{proof}
  (1)反设无零点,则$f(x)$在$(0,\pi)$不变号,而$\sin x$也不变号,因此与$\int_0^{\pi} f(x) \sin x\mathrm{d} x = 0$矛盾,因此至少有一个零点。
  若$f(x)$只有一个零点$\xi$,则$f(x)$在$(0,\xi), (\xi,\pi)$内不变号,
  由于$\int_0^{\pi} f(x) \sin x \mathrm{d} x= 0$,因此在上述两个区间符号相反,
  而$\sin(x - \xi)$也仅在$\xi$为$0$,$(0,\xi),(\xi,\pi)$符号相反,
  因此
  \begin{equation*}
    \int_0^{\pi} f(x) \sin(x - \xi)\mathrm{d} x = \cos \xi \int_0^{\pi}f(x) \sin x \mathrm{d} x - \sin \xi \int _0^{\pi} f(x) \cos x \mathrm{d} x = 0
  \end{equation*}
  这与$f(x)\sin(x - \xi)$非零矛盾,
  因此$f(x)$至少有两个零点

  (2)反设$f(x)$在$(a,b)$上至多$n$个零点,则其最多改变$n$次符号。
  由于$\int_a^b f(x)\mathrm{d}x = 0$,得到$f(x)$在$(a,b)$至少改变一次符号,
  一定有能从$n$个零点中找出$k$个零点$x_1,x_2,\cdots,x_k$满足$f(x)$在$(a,x_1),(x_1,x_2),\cdots,(x_k,b)$上不变号,
  且在相邻区间变号,
  而$g(x) = (x-x_1)(x-x_2)\cdots(x-x_k)$有同样的性质,
  于是$f(x)g(x)$不变号,
  但$\int _a^b f(x)g(x) \mathrm{d} x = 0$,
  这显然矛盾。
\end{proof}


\subsection{积分中值定理}

\begin{theorem}[积分第一中值定理]
  $f(x),g(x)$在$[a,b]$连续,$g(x)$在$[a,b]$不变号,
  则$\exists \xi \in (a,b)$使得$\int_a^b f(x)g(x)\mathrm{d}x = f(\xi) \int_a^b g(x)\mathrm{d}x$
\end{theorem}

\begin{proof}
  不妨设$g(x) \geq 0$,由于$f(x)$在$[a,b]$连续,设$M,m$分别是最大值和最小值,
  则$mg(x) \leq f(x) g(x) \leq M g(x)$,
  因此得到$m \int_a^b g(x)\mathrm{d}x \leq \int_a^b f(x)g(x) \mathrm{d}x \leq M \int_a^b g(x)\mathrm{d}x$,
  根据介值性可知(具体讨论略去)。
\end{proof}

\begin{corollary}[第一中值定理的推广]
  若$f(x),g(x)$在$[a,b]$可积(有上下确界但不一定连续),$g(x)$在$[a,b]$不变号,
  $M,m$为$f(x)$在$[a,b]$的上下确界,
  则$\exists \eta \in [m,M]$(不一定能为函数值$f(\xi)$)使得$\int_a^b f(x)g(x) \mathrm{d}x = \eta \int_a^b g(x)\mathrm{d}x$
\end{corollary}

~

\begin{exercise}[第一中值定理(技巧较强)]
  $f(x)$是$[0,1]$连续函数,$\int_0^1f(x)\mathrm{d}x = \int_0^1 xf(x)\mathrm{d}x$,
  证明:$\exists \xi \in (0,1)$使得$\int_0^{\xi}f(x)\mathrm{d}x = 0$
\end{exercise}

\begin{proof}
  考虑$F(x) = \int_0^x f(t)\mathrm{d}t$,
  想使用积分第一中值定理可证$\int_0^1 F(x)\mathrm{d}x = 0$,则可推出$F(\xi) = 0$。
  考虑$\int_0^1 F(x)\mathrm{d}x = \int_0^1 \mathrm{d}x \int_0^x f(t)\mathrm{d}t$,
  交换积分顺序得到$\int_0^1 \mathrm{d}t \int_t ^1 f(t) \mathrm{d}x = \int_0^1 (1 - t)f(t)\mathrm{d}t = 0$
\end{proof}


~

\begin{theorem}[积分第二中值定理]
  设$f(x)$在$[a,b]$可积,$g(x)$在$[a,b]$单调,则:(记忆方式:非负情况取最大)
  \begin{itemize}
  \item 若$g(x)$单增非负:$\exists \xi \in [a,b]$使得$\int_a^b f(x)g(x)\mathrm{d}x = g(b) \int_{\xi}^b f(x)\mathrm{d}x$
  \item 若$g(x)$单减非负:$\exists \xi \in [a,b]$使得$\int_a^b f(x)g(x)\mathrm{d}x = g(a) \int_a^{\xi} f(x)\mathrm{d}x$
  \item 若$g(x)$单调:$\exists \xi \in [a,b]$使得$\int_a^b f(x)g(x)\mathrm{d}x = g(a)\int_a^{\xi} f(x)\mathrm{d}x + g(b) \int_{\xi}^bf(x)\mathrm{d}x$
  \end{itemize}
\end{theorem}

\begin{proof}
  只证明(3)的特殊情况:设$f(x)$连续,$g(x)$连续可微且单调
  \begin{align*}
    \int_a^b f(x) g(x) \mathrm{d}x &= \int _a^b g(x) \mathrm{d}(F(x)) = g(x) F(x) |_a^b - \int_a^b F(x) g^{\prime}(x)\mathrm{d}x\\
                          & = g(b)F(b) - g(a) F(a) - F(\xi) \int_a^b g^{\prime}(x)\mathrm{d}x\\
                          & = g(a)[F(\xi) - F(a)] + g(b)[F(b) - F(\xi)]\\
    & = g(a) \int_a^{\xi}f(x)\mathrm{d}x + g(b) \int_{\xi}^b f(x)\mathrm{d}x
  \end{align*}
\end{proof}

\begin{note}
  第二中值定理证明过于繁琐,这里不给出全部,只给出最后一种特殊情况的证明。其最大的应用是证明反常积分的Abel和Dirichlet判别法。
\end{note}

\section{三角函数定积分计算}


\subsection{$[0,\frac{\pi}{2}]$上三角函数积分公式}

\begin{theorem}[$[0,\frac{\pi}{2}]$三角函数积分]
  设$f(x)$是连续函数,则$\int_0^{\frac{\pi}{2}}f(\sin x, \cos x)\mathrm{d}x = \int_0^{\frac{\pi}{2}}f(\cos x, \sin x)\mathrm{d}x$
\end{theorem}

\begin{proof}
  令$x = \frac{\pi}{2} - t$,
  则左侧为$\int_{\frac{\pi}{2}}^0 f(\cos t, \sin t)(-\mathrm{d}t) = \int_0^{\frac{\pi}{2}}f(\cos t, \sin t)\mathrm{d}t$
\end{proof}

~

\begin{exercise}[重要练习结论]
  (1)证明:$\forall k$,$\int_0^{\frac{\pi}{2}} \frac{\sin ^k x}{\sin^k x + \cos ^kx}\mathrm{d}x = \int_0^{\frac{\pi}{2}} \frac{\cos^k x}{\sin^k x + \cos ^k x}\mathrm{d}x = \int^{\frac{\pi}{2}}_0 \frac{x}{1 + \tan^k x} = \frac{\pi}{4}$

  (2)计算$\int_0^{\frac{\pi}{2}} \ln \tan x\mathrm{d}x$

  (3)计算$\int_0^{+\infty}\frac{\mathrm{d}x}{(1 + x^2)(1 + x^{2022})}$

  (4)计算$\int_0^{+\infty}\frac{\ln x}{4 + x^2}\mathrm{d}x$
\end{exercise}

\begin{proof}
  (1)根据对称性可知$2S = \int_0^{\frac{\pi}{2}} \mathrm{d}x = \frac{\pi}{2}$,因此$S = \frac{\pi}{4}$

  (2)$\int_0^{\frac{\pi}{2}} \ln \sin x \mathrm{d}x - \int _0^{\frac{\pi}{2}} \ln \cos x \mathrm{d}x = 0$

  (3)设$x = \tan t$,得到$\int_0^{\frac{\pi}{2}} \frac{1}{1 + \tan ^{2022}x}\mathrm{d}t$,用(1)结论可知为$\frac{\pi}{4}$

  (4)令$x = 2 \tan t$,得到$I = \int_0^{\frac{\pi}{2}} \frac{\ln (2 \tan t)}{4 \sec^2 t}2 \sec^2 t\mathrm{d}t = \frac{\pi}{4} \ln 2 + \frac{1}{2}\int_0^{\frac{\pi}{2}} \ln (\tan t)\mathrm{d}t = \frac{\pi}{4} \ln 2$
\end{proof}

~

\begin{theorem}[Wallis推广]
  $I(m,n) = \int_0^{\frac{\pi}{2}} \cos ^m x \sin ^n x \mathrm{d}x$,则$I(m,n) = I(n,m)$,且
  \begin{equation*}
    I(m,n) = \frac{m-1}{m+n} I(m-2,n) = \frac{n-1}{m+n} I(m,n-2) =
    \begin{cases}
      \frac{(m-1)!!(n-1)!!}{(m+n)!!}, &m,n\text{不全为偶数}\\
      \frac{(m-1)!!(n-1)!!}{(m+n)!!}\frac{\pi}{2},& m,n\text{均为偶数}
    \end{cases}
  \end{equation*}
\end{theorem}

~

\begin{exercise}[基础练习]
  (1)$\int_0^{\frac{\pi}{2}} \cos ^3 x \sin^2 x\mathrm{d}x$

  (2)$\int_0^{\frac{\pi}{2}} \cos ^4 x \sin^3 x \mathrm{d}x$

  (3)$\int_0^{\frac{\pi}{2}} \sin^3 \cos^5 x\mathrm{d}x$

  (4)$\int_0^{\frac{\pi}{2}}\sin^2 x \cos^4 x\mathrm{d}x$
\end{exercise}

\begin{solution}
  (1)$I(3,2) = \frac{2!!\cdot 1!!}{5!!} = \frac{2 \cdot 1}{5\cdot 3 \cdot 1} = \frac{2}{15}$

  (2)$I(4,3) = \frac{3 \cdot 2}{7\cdot 5 \cdot 3} = \frac{2}{35}$

  (3)$I(3,5) = \frac{2 \cdot 4 \cdot 2}{8\cdot 6\cdot 4\cdot 2} = \frac{1}{24}$

  (4)$I(2,4) = \frac{1 \cdot 3 \cdot 1}{6 \cdot 4 \cdot 2}\frac{\pi}{2} = \frac{\pi}{32}$
\end{solution}

\subsection{$[0,\pi],[0,2\pi]$上三角函数积分}

\begin{theorem}[对称性的积分]
  若$f(x)$关于$(x_0,0)$中心对称,则$\int_{x_0 - a}^{x_0 + a} f(x)\mathrm{d}x = 0$。
  若$f(x)$关于$x  = x_0$轴对称,则$\int_{x_0 - a}^{x_0 + a}f(x)\mathrm{d}x = 2 \int_{x_0 - a}^{x_0} f(x)\mathrm{d}x = 2 \int_{x_0}^{x_0 + a} f(x)\mathrm{d}x$
\end{theorem}

\begin{note}
  若$f(x),g(x)$都关于$x = x_0$轴对称,则$f(x)g(x)$也关于$x = x_0$轴对称。
  $f(x)$轴对称,$g(x)$中心对称,则$f(x)g(x)$中心对称。
  $f(x),g(x)$都中心对称,则$f(x)g(x)$轴对称。
\end{note}

\begin{theorem}[$[0,\pi]$上Wallis公式推广]
  对任意非负整数$m,n$,有
  \begin{equation*}
    J(m,n) = \int_0^{\pi} \sin^m x \cos ^n x\mathrm{d}x =
    \begin{cases}
      0, & n \text{为奇数}\\
      2I(m,n),&n \text{为偶数}
    \end{cases}
  \end{equation*}
  注意这里$J(m,n) \neq J(n,m)$
\end{theorem}

\begin{proof}
  根据$\sin^m x \cos^n x$在$n$为偶关于$\frac{\pi}{2}$轴对称,
  $n$为奇时中心对称。
\end{proof}

\begin{theorem}[$[0,2\pi]$上Wallis公式推广]
  对任意非负$m,n$,有
  \begin{equation*}
    K(m,n) = \int_0^{2\pi} \sin^m x \cos ^n x \mathrm{d}x =
    \begin{cases}
      0, & m,n\text{不全为偶数}\\
      4I(m,n), & m,n \text{均为偶数}
    \end{cases}
  \end{equation*}
  这里$K(m,n) = K(n,m)$
\end{theorem}

~

\begin{exercise}[$(0 \sim \pi)$上的计算]
  (1)$\int_0^{\pi} \cos^3 x \sin^2 x \mathrm{d}x$
  (2)$\int _0^{\pi} \cos^4 x \sin^3 x \mathrm{d}x$
  (3)$\int_0^{\pi} \sin^3x \cos^5 x \mathrm{d}x$
  (4)$\int_0^{\pi} \sin^2 x \cos^4 x \mathrm{d}x$
\end{exercise}

\begin{solution}
  (1)$\cos x$是奇数幂,因此为$0$

  (2)偶数幂,因此$2I(4,3) = 2 \frac{3 \cdot 2}{7 \cdot 5 \cdot 3} = \frac{4}{35}$

  (3)奇数,为$0$

  (4)偶数幂,但$m,n$均为偶数,因此结果为$\frac{\pi}{16}$
\end{solution}

~

\begin{exercise}[$(0 \sim 2\pi)$上的计算]
  (1)$\int_0^{2\pi} \cos^3 x \sin^2 x \mathrm{d}x$
  (2)$\int_0^{2\pi} \cos^2 x \sin^3 x\mathrm{d}x$
  (3)$\int_0^{2\pi}\sin^2 x \cos^4 x \mathrm{d}x$
\end{exercise}

\begin{solution}
  (1)不全为偶,因此为$0$
  (2)$0$
  (3)均为偶数,结果为$\frac{\pi}{8}$
\end{solution}

~

\begin{note}
  周期函数的积分区域可以转换至任意周期区间,因此其他区间也可计算。
\end{note}

\begin{exercise}[非标准区间]
  (1)$\int_{-\pi}^{\pi} \cos^4 x \sin^4 x \mathrm{d}x$
  (2)$\int_{-\pi}^{3\pi} \sin^5 x \cos^6 x \mathrm{d}x$
  (3)$\int_{-\pi}^0 \cos^3 x \sin^2 x \mathrm{d}x$
  (4)$\int_{-\pi}^0 \cos^4 x \sin^3 x \mathrm{d}x$
  (5)$\int_{\pi}^{2\pi}\sin^2 x \cos^4 x \mathrm{d}x$
  (6)$\int_0^{\pi} |\cos^5 x| \mathrm{d}x$
\end{exercise}

\begin{solution}
  (1)$2J(4,4) = 4I(4,4) = \frac{3}{64}\pi$

  (2)先转换到$[0,4\pi]$,再变成$2$倍的$[0,2\pi]$,
  非全为偶,因此结果为$0$

  (3)偶函数,转换为$\int_0^{\pi} \cos^3 x \sin^2 x \mathrm{d}x = 0$

  (4)奇函数,转换为$- \int_0^{\pi}\cos^4 x \sin^2 x dx = -2 I(4,3) = - \frac{4}{35}$

  (5)要看$\sin$在$x = \pi$的对称性,其是中心对称,
  因此等价于$\int_0^{\pi} \sin^2 x \cos^4 x dx = \frac{\pi}{16}$

  (6)绝对值肯定轴对称,因此$2 I(5,0) = \frac{16}{15}$
\end{solution}

~

\begin{exercise}[其他题目]
  证明:$\int_0^{2\pi}f(a \sin x + b \cos x)\mathrm{d}x = 2 \int_0^{\pi} f(\sqrt{a^2 + b^2}\cos x )\mathrm{d}x$
\end{exercise}

\begin{proof}
  根据$a \sin x + b \cos x = \sqrt{a^2 + b^2}\cos(x + \varphi)$,
  这里$\cos \varphi = \frac{b}{\sqrt{a^2 + b^2}}$。
  因此
  \begin{equation*}
    \int_0^{2\pi}f(\sqrt{a^2 + b^2} \cos(x + y))\mathrm{d}x = \int_y^{2\pi + y}f(\sqrt{a^2 + b^2}\cos t)\mathrm{d}t = \int_0^{2\pi}f(\sqrt{a^2 + b^2} \cos x)\mathrm{d}x = 2 \int_0^{\pi}f(\sqrt{a^2 + b^2} \cos x)\mathrm{d}x
  \end{equation*}
\end{proof}

~


\subsection{区间再现公式}

\begin{theorem}[区间再现]
  $f(x)$为连续函数,$\int_0^{\pi} x f(\sin x)\mathrm{d}x = \frac{\pi}{2} \int_0^{\pi} f(\sin x)\mathrm{d}x = \pi \int^{\frac{\pi}{2}}_0 f(\sin x)\mathrm{d}x$
\end{theorem}

\begin{proof}
  令$x = \pi - t$,得到$I = \int_{\pi}^0 (\pi - t)f(\sin t)(-\mathrm{d}t) = \pi \int_0^{\pi} f(\sin t)\mathrm{d}t - \int_0^{\pi}t f(\sin t)\mathrm{d}t$,
  因此得到$I = \frac{\pi}{2} \int_0^{\pi} f(\sin t)\mathrm{d}t$
\end{proof}

~

\begin{exercise}[区间再现公式的应用]
  (1)$\int_0^{\pi} x \ln \sin x \mathrm{d}x$
  (2)$\int_0^{\pi} \frac{x \sin x}{1 + \cos^2 x}\mathrm{d}x$
  (3)$\int_0^{\pi}\frac{x}{1 + \cos^2x}\mathrm{d}x$
\end{exercise}

\begin{solution}
  (1)变为$I = \frac{\pi}{2} \int_0^{\pi} \ln \sin x dx = - \frac{\pi^2}{2} \ln 2$

  (2)$I = \frac{\pi}{2} \int_0^{\pi} \frac{\sin x}{1 + \cos^2 x}dx = - \frac{\pi}{2} \arctan \cos x |^{\pi}_0 = \frac{\pi^2}{4}$

  (3)$I = \int_0^{\pi} \frac{x}{1 + \cos^2 x}dx = \frac{\pi}{2} \int_0^{\pi} \frac{1}{1 + \cos^2 x}dx = \pi \int_0^{\frac{\pi}{2}} \frac{\sec^2 x}{\sec^2 x + 1}dx = \pi \int_0^{\frac{\pi}{2}} \frac{d(\tan x)}{\tan^2 x + 2} = \frac{\pi^2}{2 \sqrt{2}}$

  (4)本题区间比较麻烦,不能用区间再现。
  可尝试用三角变换:$I = \int_0^{\frac{\pi}{4}} \frac{x}{2 \cos^2 x}dx = \frac{1}{2} \int_0^{\frac{\pi}{4}}x d(\tan x)$,
  用分部积分得到
  \begin{equation*}
    \frac{1}{2} x \tan x \bigg| _0^{\frac{\pi}{4}} + \frac{1}{2} \ln \cos x \bigg|_0^{\frac{\pi}{4}} = \frac{\pi}{8} + \frac{1}{2} \ln \frac{\sqrt{2}}{2}
  \end{equation*}
\end{solution}

\begin{note}
  上面几个虽然并非严格的$xf(\sin x)$形式,但是也可以使用区间再现。
\end{note}


\subsection{Euler积分}

\begin{exercise}[Euler积分]
  (1)Euler积分:$I = \int_0^{\frac{\pi}{2}} \ln \sin x \mathrm{d}x = - \frac{\pi}{2} \ln 2$
  (2)推论:$\int_0^{\frac{\pi}{2}} \ln \cos x \mathrm{d}x = - \frac{\pi}{2} \ln 2, \int_0^{\frac{\pi}{2}} \ln \tan x \mathrm{d}x = 0, \int_0^{\pi} \ln \sin x \mathrm{d}x = -\pi \ln 2$
\end{exercise}

\begin{solution}
  (1)该积分是瑕积分,$I = \int_0^{\frac{\pi}{2}} \ln \sin x \mathrm{d}x = \int_0^{\frac{\pi}{2}} \ln \cos x \mathrm{d}x = \frac{1}{2} \int_0^{\frac{\pi}{2}} \ln \sin x + \ln \cos x \mathrm{d}x = \frac{1}{2} \int_0^{\frac{\pi}{2}} \ln \sin x \cos x \mathrm{d}x$,
  根据倍角公式得到
  \begin{equation*}
    I = \frac{1}{2} \int_0^{\frac{\pi}{2}} \ln \frac{1}{2} \sin 2x \mathrm{d}x = \frac{1}{2} \int_0^{\frac{\pi}{2}} \ln \sin 2x \mathrm{d}x - \frac{\pi}{4} \ln 2= \frac{1}{4} \int_0^{\pi} \ln \sin t\mathrm{d}t - \frac{\pi}{4} \ln 2 = \frac{1}{2}I - \frac{\pi}{4} \ln 2
  \end{equation*}
  因此得到$I = - \frac{\pi}{2} \ln 2$
\end{solution}

\begin{note}
  要证明Euler积分收敛,只需证明$\lim \limits _{x \rightarrow 0^+} x \ln \sin x = 0$
\end{note}

~

\begin{exercise}[Euler积分的应用]
  (1)$\int_0^{\frac{\pi}{2}} x \cot x \mathrm{d}x$
  (2)$\int_0^1 \frac{\ln x}{\sqrt{1 - x^2}}\mathrm{d}x$
  (3)$\int_0^1 \frac{\arcsin x }{x}\mathrm{d}x$
  (4)$\int_0^{\pi} \frac{x \sin x}{1 - \cos x}\mathrm{d}x$
\end{exercise}

\begin{solution}
  (1)$I = \int_0^{\frac{\pi}{2}} x \mathrm{d}(\ln \sin x) = x \ln \sin x |_0^{\frac{\pi}{2}} - \int _0^{\frac{\pi}{2}} \ln \sin x \mathrm{d}x = \frac{\pi}{2} \ln 2$

  (2)令$x = \sin t$,得到$I = \int_0^{\frac{\pi}{2}} \frac{\ln \sin t}{\cos t} \cos t \mathrm{d}t = - \frac{\pi}{2} \ln 2$

  (3)令$x = \sin t$,得到$\int_0^{\frac{\pi}{2}} \frac{t}{\sin t}\cos t \mathrm{d}t = \int_0^{\frac{\pi}{2}} t \cot t \mathrm{d}t$,即(1)

  (4)注意$\cot \frac{x}{2} = \frac{\sin x }{ 1 - \cos x}$(或者可以当场对$1 - \cos x$展开),
  因此$I = \int_0^{\pi}x \cot \frac{x}{2}\mathrm{d}x = 4 \int_0^{\frac{\pi}{2}} t \cot(t)\mathrm{d}t$,变为(1)
\end{solution}

~



\subsection{Dirichlet积分}

\begin{theorem}[Dirichlet积分]
  Dirichlet积分$\int_0^{\pi} \frac{\sin \frac{2n + 1}{2}x}{2 \sin \frac{x}{2}} = \frac{\pi}{2}\mathrm{d}x$。\\
  推论:$\int_0^{\pi} \frac{\sin \frac{2n+1}{2}x}{\sin \frac{x}{2}} \mathrm{d}x = \pi, \int_0^{\frac{\pi}{2}} \frac{\sin(2n + 1)x}{\sin x}\mathrm{d} x = \frac{\pi}{2}, \int_0^{\frac{\pi}{2}} \frac{\sin(2n)x}{\sin x}\mathrm{d}x = 0, \int_0^{\pi}\frac{\sin(2n+1)x}{\sin x}\mathrm{d} x = \pi$
\end{theorem}

\begin{proof}
  (1)Dirichlet:关键在于$\cos x + \cos 2x + \cdots + \cos nx = \frac{\sin(n+\frac{1}{2})x - \sin \frac{x}{2}}{2 \sin \frac{x}{2}} = \frac{\sin \frac{2n + 1}{2}x}{2 \sin \frac{x}{2}} - \frac{1}{2}$,
  从而得到:
  \begin{equation*}
    \int_0^{\pi} \frac{\sin \frac{2n+1}{2}x}{2 \sin \frac{x}{2}} \mathrm{d}x = \sum\limits_{k = 1}^n \int_0^{\pi} \cos kx \mathrm{d}x + \frac{\pi}{2} = \sum\limits_{k = 1}^n \frac{1}{k} \sin kx \bigg|_0^{\pi} + \frac{\pi}{2} = \frac{\pi}{2}
  \end{equation*}

  (2)推论:第一个就是把分母$2$拿走,第二个(分子奇数情况)取$t = 2x$,
  第四个用对称性

  (3)
  第三个设$I_n = \int_0^{\pi} \frac{\sin 2n x}{ \sin x}\mathrm{d} x$,
  \begin{align*}
    I_n - I_{n-1} &= \int_0^{\pi}\frac{\sin 2nx - \sin (2n - 2)x}{\sin x}dx = \int_0^{\pi} \frac{2}{\sin x}(\cos \frac{4n - 2}{2}x \cdot \sin \frac{2n-(2n-2)}{2}x)\mathrm{d} x\\
   & = 2 \int_0^{\pi} \cos (2n - 1)x \mathrm{d} x = \frac{2}{2n - 1}\sin (2n - 1)x \bigg|^{\pi}_0 = 0 
  \end{align*}
  而$I_1 = 0$,因此$I_n = 0$
\end{proof}


\subsection{其他几个常见三角积分}

\begin{exercise}[几个常见三角分数积分]
  (1)$\int_0^{\pi} \frac{dx}{1 + \sin^2 x}dx $
  (2)$\int_0^{2\pi} \frac{dx}{1 + a \cos x}, (-1 < a < 1)$
\end{exercise}

\begin{solution}
  (1)上下同乘$\sec^2 x$,得到$\int \frac{\sec^2 x}{\sec^2 x + \tan ^2 x}dx = \int \frac{d(\tan x)}{2 \tan^2 x + 1} = \frac{1}{\sqrt{2}} \arctan (\sqrt{2} \tan x) + c$,
  注意$\tan x$在$\frac{\pi}{2}$是间断的!
  根据对称性转换为$I = 2 \int_0^{\frac{\pi}{2}} \frac{dx}{1 + \sin^2 x} = \sqrt{2} \arctan (\sqrt{2} \tan x)|^{\frac{\pi}{2}}_0 = \sqrt{2}(\frac{\pi}{2} - 0) =  2\frac{\pi}{\sqrt{2}}$

  (2)具体的不定积分计算见不定积分一节,是非常经典的积分。
  $I = 2 \int_0^{\pi} \frac{dx}{1 + a \cos x} = \frac{4}{\sqrt{1 - a^2}} \arctan \sqrt{\frac{1 - a}{1 + a}}\tan \frac{x}{2}|_0^{\pi}$,
  显然$x = \pi$为间断点,使用对称性非常正确!
  因此结果为$\frac{2\pi}{\sqrt{1 - a^2}}$
\end{solution}

\begin{note}
  一定要注意函数的连续性,用对称性转换。因此能用对称性的一定要优先用对称性。
  上面两个题目也是重要的结论。
\end{note}

~



\section{积分(不)等式}

\subsection{整体放缩}

\begin{exercise}[几个简单积分放缩]
  证明(1)$1 < \int_0^{\frac{\pi}{2}} \frac{\sin x}{x} \mathrm{d} x < \frac{\pi}{2}$
  (2)$\frac{2}{9}\pi^2 < \int_{\frac{\pi}{6}}^{\frac{\pi}{2}} \frac{2x}{ \sin x}\mathrm{d} x < \frac{1}{3}\pi^2$
  (3)$3 \sqrt{e} < \int_e^{4e} \frac{\ln x}{\sqrt{x}} < 6$
\end{exercise}

\begin{proof}
  (1)根据$f^{\prime}(x) = \frac{x - \tan x}{x^2 \sec x }$可知函数单减,
  因此
  \begin{equation*}
    f \left( \frac{\pi}{2} \right) = \frac{2}{\pi} \leq \frac{\sin x}{x} < f(0) = 1
  \end{equation*}

  (2)根据导数可知单调增,因此
  \begin{equation*}
    f \left( \frac{\pi}{6} \right) = \frac{2}{3}\pi \leq \frac{2x}{\sin x} \leq f \left( \frac{\pi}{2} \right) = \pi
  \end{equation*}

  (3)$f^{\prime}(x) = \frac{2 - \ln x}{2x \sqrt{x}}$,
  因此$\frac{1}{\sqrt{e}} = f(e) \leq f(x) \leq f(e^2) = \frac{2}{e}$
\end{proof}


\subsection{同区域单调性放缩}

\begin{exercise}[几道经典同区域相减]
  (1)$f(x)$在$[a,b]$上严格增的连续函数,
  证明:$\int_a^b x f(x)\mathrm{d} x > \frac{a+b}{2} \int_a^b f(x)\mathrm{d} x$

  (2)证明$\int_0^{\frac{\pi}{2}} \frac{\cos x}{1 + x^2}\mathrm{d} x > \int_0^{\frac{\pi}{2}} \frac{\sin x}{1 + x^2}\mathrm{d} x$
  (3)证明$\int_0^{\sqrt{2\pi}} \sin x^2 \mathrm{d} x > 0$
\end{exercise}

\begin{proof}
  (1)等价于证明$\int_a^b \left( x - \frac{a+b}{2} \right)f(x) \mathrm{d} x > 0$,
  进行有目的性的换元,分为两段,用$t$表示$x$与$\frac{a+b}{2}$之间的距离:
  \begin{align*}
    \int_a^b \left( x - \frac{a+b}{2} \right)f(x)\mathrm{d} x &=  \int_a^{\frac{a+b}{2}} \left( x - \frac{a+b}{2} \right) f(x)\mathrm{d} x + \int_{\frac{a+b}{2}}^b \left( x - \frac{a+b}{2} \right) f(x)\mathrm{d} x\\
                                                              &= \int_{\frac{b-a}{2}}^0 (-t) f(\frac{a+b}{2} - t)(-\mathrm{d} t) + \int_0^{\frac{b-a}{2}} tf(t + \frac{a+b}{2})\mathrm{d} t\\
    &= \int _0^{\frac{b-a}{2}} t \left[ f(t + \frac{a+b}{2}) - f(\frac{a+b}{2} - t) \right]\mathrm{d} t > 0
  \end{align*}

  (2)等价于$\int_0^{\frac{\pi}{2}} \frac{\sin x - \cos x}{1 + x^2}\mathrm{d} x < 0$,
  而$\sin x- \cos x = \sqrt{2} \sin (x - \frac{\pi}{4})$关于$\left( \frac{\pi}{4},0 \right)$中心对称,
  且$\frac{1}{1+x^2}$单减。
  因此令$u = x - \frac{\pi}{4}$:
  \begin{align*}
   \int_0^{\frac{\pi}{2}} \frac{\sin x - \cos x }{1 + x^2}\mathrm{d} x & = \int _0^{\frac{\pi}{2}} \frac{\sqrt{2} \sin (x - \frac{\pi}{4})}{1 + x^2}\mathrm{d} x = \sqrt{2} \int_{-\frac{\pi}{4}}^{\frac{\pi}{4}} \int_{-\frac{\pi}{4}}^{\frac{\pi}{4}} \frac{\sin u}{1 + (\frac{\pi}{4} + u)^2}\mathrm{d} u\\
    &= \sqrt{2} \left( \int_0^{\frac{\pi}{4}} \frac{\sin x}{1 + (\frac{\pi}{4} + x)^2} \mathrm{d} x - \int _0^{\frac{\pi}{4}} \frac{\sin x}{1 + (\frac{\pi}{4} - x)^2}\mathrm{d} x \right)\\
                                                                       &= \sqrt{2} \int_0^{\frac{\pi}{4}} \left( \frac{1}{(\frac{\pi}{4} + x)^2} - \frac{1}{(\frac{\pi}{4} - x)^2} \right)\sin x \mathrm{d} x< 0
  \end{align*}
\end{proof}

~

\begin{exercise}[一道特殊题]
  (1)证明$\int_0^1 \frac{\cos x}{\sqrt{1 - x^2}}\mathrm{d} x > \int_0^1 \frac{\sin x}{\sqrt{1 - x^2}}\mathrm{d} x$
\end{exercise}


\subsection{利用Newton-Leibniz公式}

\begin{theorem}[Newton-Leibniz放缩]
  若$f(x)$为$[a,b]$上连续可微函数,则对$\forall x_1,x_2 \in [a,b]$有$f(x_2) = f(x_1) + \int_{x_1}^{x_2}f^{\prime}(x)\mathrm{d} x$,因此
  \begin{equation*}
    |f(x_2)| \leq |f(x_1)| + \int_a^b |f^{\prime}(x)|\mathrm{d} x
  \end{equation*}
\end{theorem}

~

\begin{exercise}[几道经典题]
  下面均假设$f(x)$在区间上连续可微

  (1)证明$|f(0)| \leq \frac{1}{a} \int_0^a |f(x)|\mathrm{d} x + \int_0^a |f^{\prime}(x)| \mathrm{d} x$

  (2)经典:证明$|f(x)| \leq \left| \frac{1}{b-a}\int_a^b f(x)\mathrm{d}x \right| + \int_a^b |f^{\prime}(x)| \mathrm{d} x$

  (3)证明$|f(x)| \leq \int_0^1 (|f(x)| + |f^{\prime}(x)|)\mathrm{d} x$

  (4)重点:证明$\left| f \left( \frac{1}{2} \right) \right| \leq \int_0^1|f(x)| \mathrm{d} x + \frac{1}{2} \int_0^1 |f^{\prime}(x)| \mathrm{d} x$
\end{exercise}

\begin{proof}
  (1)用积分第一中值定理可知$\frac{1}{a}\int_0^a |f(x)|\mathrm{d} x = |f(\xi)|$,因此得证

  (2)(3)同理

  (4)设$|f(x_0)| = \min \limits_{x \in [0,\frac{1}{2}]} |f(x)|$,
  则$|f(x_0)| = 2 \int_0^{\frac{1}{2}}|f(x_0)|\mathrm{d} x \leq 2 \int_0^{\frac{1}{2}} |f(x)|\mathrm{d} x$,
  因此
  \begin{equation*}
      \left| f \left( \frac{1}{2} \right) \right| = \left| f(x_0) + \int_{x_0}^{\frac{1}{2}}f^{\prime}(x)\mathrm{d} x \right| \leq 2 \int_0^{\frac{1}{2}} |f(x)|\mathrm{d} x + \int_0^{\frac{1}{2}} |f^{\prime}(x)|\mathrm{d} x\\
  \end{equation*}
  同理取$[\frac{1}{2},1]$的部分可得到
  \begin{equation*}
    \left| f \left( \frac{1}{2} \right) \right| \leq 2 \int_{\frac{1}{2}}^1 |f(x)|\mathrm{d} x + \int_{\frac{1}{2}}^1 |f^{\prime}(x)|\mathrm{d} x
  \end{equation*}
  因此两式相加除以$2$即得到结果。
\end{proof}

\subsection{利用Cauchy-Schwarz不等式}

\begin{theorem}[Cauchy-Schwarz不等式]
  设$f(x),g(x)$在$[a,b]$上可积,则
  \begin{equation*}
    \left( \int_a^b f(x)g(x)\mathrm{d} x \right)^2 \leq \int_a^b f^2(x)\mathrm{d} x \int_a^b g^2(x)\mathrm{d} x
  \end{equation*}
\end{theorem}

\begin{corollary}
  若$f(x),g(x)$非负可积,则
  \begin{equation*}
    \int_a^b f(x)\mathrm{d} x \int_a^b g(x)\mathrm{d} x \geq \left( \int_a^b \sqrt{f(x)g(x)}\mathrm{d} x \right)^2
  \end{equation*}
\end{corollary}

~

\begin{exercise}[一些经典题目]
  (1)$f(x)$在$[0,1]$连续可微,
  $f(1) = f(0) = 1$,证明$\int_0^1 [f^{\prime}(x)]^2 \mathrm{d} x \geq 1$
\end{exercise}

\begin{proof}
  (1)直接根据Cauchy-Schwarz不等式得到$\int_0^1 [f^{\prime}(x)]^2 \mathrm{d} x \geq \left[ \int_0^1 f^{\prime}(x)\mathrm{d} x \right]^2 \geq \frac{1}{4}$
\end{proof}

\subsection{利用微分中值定理}

\begin{exercise}[几道具体分段题]
  (1)一阶导:$f(x)$在$[0,2]$上可导,
  $f(0) = f(2) = 1$,$|f^{\prime}(x)| \leq 1$,
  证明$1 < \int_0^2 f(x)\mathrm{d} x < 3$

  (2)二阶导:
\end{exercise}

\begin{proof}
  (1)左侧:进行分段处理,对于$[0,1],[1,2]$分别有
  \begin{equation*}
    \begin{cases}
      f(x) = f(0) + f^{\prime}(\xi) x \geq 1 - x , & x \in [0,1]\\
      f(x) = f(2) + f^{\prime}(\eta)(x-2) \geq x - 1, & x \in [1,2]
    \end{cases}
  \end{equation*}
  因此$\int_0^2 f(x)\mathrm{d} x \geq \int_0^1 (1 - x)\mathrm{d} x + \int_1^2 (x - 1)\mathrm{d} x = 1$,
  且等号取到时在$1$处不可导,故严格。
  同理可证明$f(x) \geq h(x)$,其中$h(x)$为
  \begin{equation*}
    h(x) =
    \begin{cases}
      1 + x, & x \in [0,1]\\
      3 - x, & x \in [1,2]
    \end{cases}
  \end{equation*}
  因此$\int_0^2 f(x)\mathrm{d} x \leq 3$
\end{proof}

~

\begin{exercise}[利用一阶导数研究积分]
  $f(x)$在$[a,b]$连续可微,$M = \max \limits_{x \in [a,b]}|f^{\prime}(x)|$,证明
  
  (1)若$f(a) = 0$或$f(b) = 0$,则$\left| \int_a^b f(x)\mathrm{d} x \right| \leq \int_a^b |f(x)| \mathrm{d} x \leq \frac{M}{2}(b-a)^2$

  (2)若$f(a) = f(b) = 0$,则$\left| \int_a^b f(x)\mathrm{d} x \right| \leq \int_a^b |f(x)|\mathrm{d} x \leq \frac{M}{4}(b-a)^2$

  (3)若$f \left( \frac{a+b}{2} \right) = 0$
\end{exercise}

\begin{proof}
  (1)设$f(a) = 0$,则$f(x) = f^{\prime}(\xi)(x-a)$,
  因此$\int_a^b |f(x)|\mathrm{d} x \leq M \int_a^b (x - a)\mathrm{d} x = \frac{M}{2}(b-a)^2$

  (2)若$f(a) = f(b) = 0$,则分两段在$a,b$展开
  \begin{equation*}
    \begin{cases}
      \int_a^{\frac{a+b}{2}}|f(x)|\mathrm{d} x \leq \frac{M}{2} \cdot \frac{(b-a)^2}{4}\\
      \int_{\frac{a+b}{2}}^b |f(x)|\mathrm{d} x \leq \frac{M}{2} \cdot \frac{(b-a)^2}{4}
    \end{cases}
  \end{equation*}

  (3)若$f \left( \frac{a+b}{2} \right) = 0$,则分两段在$f \left( \frac{a+b}{2} \right)$展开,和(2)类似可得到结果
\end{proof}

~

\begin{exercise}[利用二阶导数研究积分]
  $f(x)$在$[a,b]$二阶可微,$|f^{\prime\prime}(x)| \leq M$,证明

  (1)若$f \left( \frac{a+b}{2} \right) = 0$,则$\left| \int_a^b f(x)\mathrm{d} x \right| \leq \frac{M}{24}(b-a)^3$

  (2)若$f(a) = f(b) = 0$,则$\left| \int_a^b f(x)\mathrm{d} x \right| \leq \frac{M}{12}(b-a)^3$
\end{exercise}

\begin{proof}
  (1)在$\frac{a+b}{2}$处展开:
  \begin{equation*}
    f(x) = f \left( \frac{a+b}{2} \right) + f^{\prime}\left( \frac{a+b}{2} \right) \left( x - \frac{a+b}{2} \right) + \frac{f^{\prime\prime}(\xi)}{2} \left( x - \frac{a+b}{2} \right)^2
  \end{equation*}
  显然一阶导数项中心对称,故消去,二阶导数项用积分第一中值定理可以得到:
  $\int _a^b f(x)\mathrm{d} x = (b - a)f \left( \frac{a+b}{2} \right) + \frac{f^{\prime\prime}(\eta)}{24}(b-a)^3$,经过放缩即结论。
\end{proof}


% \subsection{使用凹凸性}


% \section{积分学的应用}

% \subsection{计算平面图形面积}

% \begin{theorem}[一般形式平面图形面积]
%   介于$x=a,x=b$以及$y_1 = f(x),y_2 = g(x)$之间的图形面积为
%   \begin{equation*}
%     S = \int _a^b (f(x) - g(x))dx
%   \end{equation*}
% \end{theorem}

% \begin{theorem}[极坐标平面图形面积]
%   曲线$\Gamma: r = r(\theta)$与射线$\theta = \alpha, \theta = \beta$夹成的
%   图形面积为:
%   \begin{equation*}
%     S = \frac{1}{2}\int _{\alpha}^{\beta}r^2(\theta)d\theta
%   \end{equation*}
% \end{theorem}

% \begin{example}
%   计算双纽线$r^2 = a^2 \cos 2\theta$围成的面积
% \end{example}

% \begin{solution}
%   首先计算$\theta$的范围,由于$a^2 \cos 2\theta \geq 0$,
%   因此$\theta \in [-\frac{\pi}{4},\frac{\pi}{4}] \cup [\frac{3\pi}{4}, \frac{5\pi}{4}]$,
%   再根据对称性等价于
%   \begin{equation*}
%     \int^{\frac{\pi}{4}}_{-\frac{\pi}{4}} r^2d\theta = \frac{a^2}{2}\int ^{\frac{\pi}{2}}_{-\frac{\pi}{2}}\cos \theta d\theta = a^2
%   \end{equation*}
% \end{solution}