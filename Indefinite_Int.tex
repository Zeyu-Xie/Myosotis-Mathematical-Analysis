
\chapter{不定积分}


\section{常用不定积分公式}

必背部分:

\begin{equation*}
  \begin{array}{lll}
    \text{普通三角:}& \int \sin x \mathrm{d}x = - \cos x + c& \int \cos x \mathrm{d}x = \sin x + c\\
    \text{sec、csc:}                   & \int \sec^2 x \mathrm{d}x = \tan x + c& \int \csc^2 x \mathrm{d}x = - \cot x + c\\
    & \int \sec x \mathrm{d} x = \ln|\sec x + \tan x| & \int \sec^3 x \mathrm{d} x = \frac{1}{2}\tan x \sec x + \frac{1}{2}\ln |\tan x + \sec x|\\
    \text{乘积三角:} & \int \sec x \tan x \mathrm{d}x = \sec x + c& \int \csc x\cot x \mathrm{d}x = - \csc x + c\\
    \text{分式:}& \int \frac{1}{x^2 + a^2} \mathrm{d}x = \frac{1}{a} \arctan \frac{x}{a} + c & \int \frac{1}{x^2 - a^2}\mathrm{d}x = \frac{1}{2a} \ln |\frac{x-a}{x+a}| + c\\
   & \int \frac{1}{\sqrt{x^2 + a^2}} \mathrm{d}x = \ln |x + \sqrt{x^2 + a^2}|+c& \int \frac{1}{\sqrt{x^2 - a^2}} \mathrm{d}x = \ln |x + \sqrt{x^2 - a^2}| + c \\
   & \int \frac{1}{\sqrt{a^2 - x^2}}\mathrm{d}x = \arcsin \frac{x}{a} + c& \\
    \text{对数:}&\int \ln x \mathrm{d} x = x \ln x - x + c&\int \ln(1 + x)\mathrm{d} x = (1+x) \ln(1+x) - x + c \\
  \end{array}
\end{equation*}

会算部分:

\begin{equation*}
  \begin{array}{lll}
    \text{三角:}& \int \tan x \mathrm{d}x = -\ln |\cos x| + c & \int \cot x \mathrm{d}x = \ln |\sin x| + c\\
    \text{平方三角:}& \int \sin^2 x \mathrm{d}x = \frac{x}{2} - \frac{\sin 2x}{4} + c& \int \cos^2 x \mathrm{d}x = \frac{x}{2} + \frac{\sin 2x}{4} + c\\
                 & \int \tan^2 x \mathrm{d}x = \tan x - x + c & \int \cot^2 x \mathrm{d}x = -\cot x - x + c\\
    \text{复合三角:}& \int e^{ax} \cos bx \mathrm{d} x = \frac{e^{ax}}{a^2 + b^2}(a \cos bx + b \sin bx) + c&\\
                 & \int e^{ax} \sin bx \mathrm{d} x = \frac{e^{ax}}{a^2 + b^2} (a \sin bx - b \cos bx) + c&\\
                 & \int x \cos nx \mathrm{d} x = \frac{1}{n^2} \cos nx + \frac{x}{n} \sin nx + c&\\
                 & \int x \sin nx \mathrm{d} x = \frac{1}{n^2} \sin nx - \frac{x}{n} \cos nx + c&\\
    \text{根式:}&\int \sqrt{x^2 \pm a^2}\mathrm{d} x = \frac{1}{2} \left[ x \sqrt{x^2 \pm a^2} \pm a^2 \ln|x + \sqrt{x^2 \pm a^2}| \right] + c&\\
                 &\int \sqrt{a^2 - x^2}\mathrm{d} x = \frac{1}{2} \left[ x \sqrt{a^2 - x^2} + a^2 \arcsin \frac{x}{a}  \right] + c&
  \end{array}
\end{equation*}



\section{不定积分基本方法}

\begin{theorem}[分部积分公式]
  函数$u,v$的积分满足:
  \begin{equation*}
    \int u \mathrm{d} v = uv - \int v \mathrm{d} u
  \end{equation*}
\end{theorem}

\begin{theorem}[表格法]
  函数$u,v$的积分$\int uv^{(n+1)} \mathrm{d} x = uv^{(n)} - u^{\prime}v^{(n-1)} + u^{\prime\prime}v^{(n-2)}- \cdots + (-1)^n u^{(n)}v +(-1)^{n+1}\int u^{(n+1)} v\mathrm{d}x$,具体操作时可列以下表格:
  \begin{center}
    \begin{tabular}[htp]{|c|c|c|c|c|c|c|}
      \hline
      $u$的各阶导数&$u$&$u^{\prime}$&$u^{\prime\prime}$&$u^{\prime\prime\prime}$&$\cdots$&$u^{(n+1)}$\\
      \hline
      $v^{(n+1)}$的各阶原函数&$v^{(n+1)}$&$v^{(n)}$&$v^{(n-1)}$&$v^{(n-2)}$&$\cdots$&$v$\\
      \hline
    \end{tabular}
  \end{center}
  计算方法:以$u$为起点,左上右下相乘,符号先正后负交替,最后一项为最后一列两个的积分$(-1)^{n+1}\int u^{(n+1)}v \mathrm{d} x$
\end{theorem}

\begin{note}
  $\int P_n(x) e^{kx}\mathrm{d} x, \int P_n(x) \sin ax \mathrm{d} x, \int P_n(x) \cos ax \mathrm{d} x$都可以使用表格法快速算出
\end{note}

~

\begin{exercise}[$P_n(x)e^{kx}$型]
  计算$\int(x^3 + 2x + 6)e^{2x}\mathrm{d} x$
\end{exercise}

\begin{solution}
  先列表格:
  \begin{center}
    \begin{tabular}[htp]{|c|c|c|c|c|}
      \hline
      $x^3+2x+6$& $3x^2 + 2$& $6x$&$6$&$0$\\
      \hline
      $e^{2x}$&$\frac{1}{2}e^{2x}$&$\frac{1}{4}e^{2x}$&$\frac{1}{8}e^{2x}$&$\frac{1}{16}e^{2x}$\\
      \hline
    \end{tabular}
  \end{center}
  得到结果为$(x^3 + 2x + 6)\frac{1}{2}e^{2x} - (3x^2 + 2)\frac{1}{4}e^{2x} + 6x(\frac{1}{8}e^{2x}) - 6 (\frac{1}{16}e^{2x}) + \int 0 \cdot (\frac{1}{16}e^{2x}) \mathrm{d} x$
\end{solution}


\begin{theorem}[积分有理分解]
  若被积函数为$R(x) = \frac{P(x)}{Q(x)}$,且$Q(x)$的次数大于等于$3$,
  若$Q(x) = (x - a_1)^{\lambda_1}\cdots (x - a_k)^{\lambda_k}(x^2 + p_1x + q_1)^{\mu_1}\cdots (x^2 + p_lx + q_l)^{\mu_l}$,
  对于$(x - a_1)^{\lambda_1}$和$(x^2 + p_1x + q_1)^{\mu_1}$项,对应展开式为:
  \begin{equation*}
    \begin{cases}
      \frac{A_1}{x - a_1} + \frac{A_2}{(x - a_1)^2} + \cdots + \frac{A_{\lambda_1}}{(x - a)^{\lambda_1}}\\
      \frac{B_1x + C_1}{x^2 + p_1x + q_1} + \cdots + \frac{B_{\mu_1}x + C_{\mu_1}}{(x^2 + p_1x + q_1)^{\mu_1}}
    \end{cases}
  \end{equation*}
  其中$\frac{A_{\lambda_l}}{(x - a)^{\lambda_l}}$的积分好算,
  而$\frac{B_{\mu_l}x + C_{\mu_l}}{(x^2 + p_1x + q_1)^{\mu_l}}$可配方化为$\int \frac{Lu + N}{(u^2 + r^2)^j}\mathrm{d} u$(因为判别式一定小于$0$),
  第一部分$L \int \frac{u}{(u^2 + r^2)^j}\mathrm{d} u$好算,
  第二部分$N \int \frac{1}{(u^2 + r^2)^j}\mathrm{d} u$用$u = r \tan v$替换,
  变为形如$\int \cos^k v \mathrm{d} v$的不定积分,有递推公式辅助求解。
\end{theorem}

~

\begin{theorem}[三角有理式转化为有理式]
  若$R(\sin x, \cos x)$为关于$\sin x, \cos x$的有理式,
  记$t = \tan \frac{x}{2}$,
  则$x = 2 \arctan t, \mathrm{d} x = \frac{2}{1+t^2}\mathrm{d} t$,
  以及万能公式
  \begin{equation*}
    \sin x = \frac{2t}{1 + t^2}, \cos x = \frac{1 - t^2}{1+t^2}, \tan x = \frac{2t}{1 - t^2}
  \end{equation*}
  因此
  \begin{equation*}
    \int R(\sin x, \cos x )\mathrm{d} x = \int R \left( \frac{2t}{1+t^2}, \frac{1 - t^2}{1 + t^2} \right) \frac{2}{1+t^2} \mathrm{d} t
  \end{equation*}
\end{theorem}

~

\begin{theorem}[两类无理根式的不定积分]
  对于$\int R \left( x, \sqrt[n]{\frac{ax + b}{cx + d}} \right) \mathrm{d} x$,只需要令$t = \sqrt[n]{\frac{ax + b}{cx + d}}$即可化为有理函数。
  对于$\int R(x, \sqrt{ax^2 + bx + c})$,要么进行配方,要么用Euler变换。
\end{theorem}

\section{不定积分表选例}

下面假设必背部分都会背,且不做计算证明

\subsection{三角函数:重点}

\begin{exercise}[核心会算]
  \begin{equation*}
    \begin{array}{ll}
      \int \tan x \mathrm{d} x = - \ln |\cos x| + c&\int \cot x \mathrm{d} x =  \ln |\sin x| + c\\
      \int \sec x\mathrm{d} x = \ln |\sec x + \tan x| + c & \int \csc x \mathrm{d} x = - \ln |\csc x + \cot x| + c
    \end{array}
  \end{equation*}
\end{exercise}

\begin{solution}
  $\int \tan x \mathrm{d} x = - \int \frac{\mathrm{d} \cos x}{\cos x}$,$\int \cot x \mathrm{d} x$同理。
  另外两个要记住
  \begin{equation*}
    \int \sec x \mathrm{d} x = \int \frac{\sec x ( \sec x + \tan x)}{\sec x + \tan x}\mathrm{d} x  = \int \frac{\mathrm{d}(\sec x + \tan x)}{\sec x + \tan x}= \ln |\sec x + \tan x|+c
  \end{equation*}
\end{solution}

~

\begin{exercise}[简单平方]
  (1)$\int \sin^2 x \mathrm{d} x$(2)重点:$\int \tan^2 x\mathrm{d} x$
  (3)超重点:$\int \sec^3 x \mathrm{d} x$
\end{exercise}

\begin{solution}
  (1)直接用半角公式

  (2)这个要记住,$\int \tan^2 x \mathrm{d} x = \int \sec^2 x - 1 \mathrm{d} x = \tan x - x + c$

  (3)这个经常会用到!
  \begin{align*}
    \int \sec^3x \mathrm{d} x &= \int \sec x \mathrm{d} \tan x = \sec x \tan x - \int \tan^2 x \sec x \mathrm{d} x\\
    &= \sec x \tan x - \int \sec^3 x \mathrm{d} x + \int \sec x \mathrm{d} x
  \end{align*}
  因此$I = \frac{1}{2} \sec x \tan x + \ln |\sec x + \tan x| + c$
\end{solution}

~


\begin{exercise}[$\sin x, \cos x$加常数分式积分]
  (1)重点:$\int \frac{\mathrm{d} x}{1 \pm \cos x}$(2)重点:$\int \frac{\mathrm{d} x}{1 \pm \sin x}$

  (3)极度重点,含参积分大量用到:$\int \frac{\mathrm{d} x}{1 + a\cos x}, |a| \leq 1$
\end{exercise}

\begin{solution}
  (1)根据$1 + \cos x = 2 \cos^2 \frac{x}{2}, 1 - \cos x = 2 \sin^2 \frac{x}{2}$,因此
  \begin{equation*}
    \begin{cases}
      \int \frac{\mathrm{d} x}{1 + \cos x} = \frac{1}{2}\int \sec^2 \frac{x}{2}\mathrm{d} x = \tan \frac{x}{2} + c\\
      \int \frac{\mathrm{d} x }{1 - \cos x} = \frac{1}{2} \int \csc^2 \frac{x}{2} \mathrm{d} x = -\cot \frac{x}{2} + c
    \end{cases}
  \end{equation*}

  (2)根据诱导公式$\sin x = \cos \left( \frac{\pi}{2} - x \right)$
  \begin{equation*}
    \begin{cases}
      \int \frac{\mathrm{d} x }{1 + \sin x} = \int \frac{\mathrm{d} x }{1 + \cos \left( \frac{\pi}{2} - x \right)} = -\tan \left( \frac{\pi}{4} - \frac{x}{2} \right)+c\\
      \int \frac{\mathrm{d} x}{1 - \sin x} = \int \frac{\mathrm{d} x}{1 - \cos \left( \frac{\pi}{2} - x \right)} = \cot \left( \frac{\pi}{4} - \frac{x}{2} \right) + c
    \end{cases}
  \end{equation*}

  (3)$a = \pm 1$时即(1)。
  用万能公式,令$t = \tan \frac{x}{2}$,
  则$\mathrm{d} x = \frac{2}{1+t^2}\mathrm{d} t$,
  因此
  \begin{equation*}
    I = \int \frac{1}{1 + a \frac{1 - t^2}{1+t^2}} \cdot \frac{2}{1+t^2}\mathrm{d} t = \int \frac{2}{(1 + t^2) + a(1 - t^2)}\mathrm{d} t = \int \frac{2}{(1-a)t^2 + (1 + a)}\mathrm{d} t = \frac{2}{1-a} \sqrt{\frac{1-a}{1+a}} \arctan \left( \sqrt{\frac{1-a}{1+a}}t \right)+ c
  \end{equation*}
  得到结果为$I = \frac{2}{\sqrt{1 - a^2}} \arctan \left( \sqrt{\frac{1-a}{1+a}} \tan \frac{x}{2} \right) + c$
\end{solution}

~


~

\begin{theorem}[$\frac{1}{a\sin^2x + b \cos^2 x}$积分]
  形如$\int \frac{\mathrm{d} x}{a \sin^2 x + b \cos^2 x}$的积分都采用上下同除以$\cos^2 x$,
  转换为
  \begin{equation*}
    \int \frac{\sec^2 x \mathrm{d} x}{a \tan^2 x + b}
  \end{equation*}
  直接用$\mathrm{d}(\tan x )= \sec^2 x \mathrm{d}x$即可。
\end{theorem}

\begin{theorem}[$\frac{1}{a\sin^2x + b\sin x \cos x+ c \cos^2 x + d}$积分]
  形如$\int \frac{\mathrm{d} x}{a \sin^2 x + b \sin x\cos x + c \cos^2 x + d}$的积分首先上下同除以$\cos^2 x$,
  得到
  \begin{equation*}
    \int \frac{\sec^2 x \mathrm{d} x}{a \tan^2 x + b\tan x + c + d \sec^2 x}
  \end{equation*}
  再利用$\sec^2 x = \tan^2 x + 1$,全部转换为$\tan x$,再分母配方即可
\end{theorem}

~


\begin{exercise}[$\sin x, \cos x$无常数、无分子分式]
  (1)重点:$\int \frac{\mathrm{d} x}{\sin^2 x + 2 \cos ^2 x}$
  (2)$\int \frac{1}{\sin^2 x \cos x}\mathrm{d} x$
  (3)$\int \frac{\mathrm{d} x}{\sin x \cos^2 x}$
\end{exercise}

\begin{solution}
  (1)同除以$\cos^2 x$,得到$\int \frac{\sec^2 x \mathrm{d} x}{2 + \tan^2 x}$,易算

  (2)同乘$\cos x$,得到$I = \int \frac{\mathrm{d}(\sin x)}{\sin^2x (1 - \sin^2 x)}$,
  即$\int \frac{1}{\sin^2 x}\mathrm{d} (\sin x) + \int \frac{1}{1 - \sin^2 x}\mathrm{d} x$,
  前者直接算,后者用$\int \frac{1}{x^2 - a}\mathrm{d} x = \frac{1}{2a} \ln \left| \frac{x-a}{x+a} \right|$。

  (3)与(2)同理
\end{solution}
~


\begin{exercise}[$\sin x, \cos x$的高次]
  (1)$\int (\sin^4 x  + \cos^4 x)\mathrm{d} x$
  (2)重点:$\int \frac{\sin x \cos x}{\sin^4 x + \cos^4 x}\mathrm{d} x$
  (3)重点:$\int \frac{\mathrm{d} x}{\sin^4x + \cos^4 x}$

  (4)$\int \sin^6 x + \cos^6 x \mathrm{d} x$
  (5)$\int \frac{\mathrm{d} x}{\sin^6 x + \cos^6 x}$
\end{exercise}

\begin{solution}
  (1)降到一次:$\sin^4 x + \cos^4 x = (\sin^2 x + \cos^2 x)^2 - 2 \sin^2 x \cos^2 x = 1 - \frac{1}{2}\sin^2 2x = 1 - \frac{1}{4}(1 - \cos 4x)$

  (2)降到二次,并切换$\sin x, \cos x$:$\int \frac{\frac{1}{2}\sin 2x}{1 - \frac{1}{2} \sin^2 2x}$,
  要善于切换正余弦:
  \begin{equation*}
    \int \frac{\sin 2x}{2 - \sin^2 2x} \mathrm{d} x = -\frac{1}{2}\int \frac{\mathrm{d}(\cos 2x)}{1 + \cos^2 2x} = - \frac{1}{2}\arctan (\cos 2x) + c
  \end{equation*}

  (3)直接降到二次,并转换为$\tan x, \sec x$:使用$\sec^2 x = 1 + \tan ^2 x$
  \begin{equation*}
    I = \int \frac{\mathrm{d} x}{1 - \frac{1}{2}\sin^22x} = \int \frac{\sec^2 2x}{\sec^2 2x - \frac{1}{2} \tan^2 2x} = \frac{1}{2} \int \frac{\mathrm{d}(\tan 2x)}{1 + \frac{1}{2} \tan^2 2x} = \frac{1}{\sqrt{2}} \arctan \frac{\tan 2x}{\sqrt{2}} + c
  \end{equation*}

  (4)$\sin^6 x + \cos^6 x = (\sin^2 x + \cos^2 x)(\sin^4 - \sin^2 \cos^2 x + \cos^4 x)$转化为四次,
  再配方:
  \begin{equation*}
    (\sin^2 x + \cos^2 x)^2 - 3 \sin^2 x \cos^2 x = 1 - \frac{3}{4} \sin^2 2x = 1 - \frac{3}{8} \left( 1 - \cos 4x \right)
  \end{equation*}
  结果为$\frac{5}{8}x + \frac{3}{32} \sin 4x + c$

  (5)$I = \int \frac{1}{1 - \frac{3}{4} \sin^2 2x}\mathrm{d} x = \int \frac{\sec^2 x}{\sec^2 2x - \frac{3}{4}\tan^2 2x}\mathrm{d} x = 2 \int \frac{\mathrm{d}(\tan 2x)}{\tan^2 2x + 4}$
\end{solution}


\begin{note}
  高次问题,如果分母为$1$则降到一次直接积分,
  分子为$1$则降到二次转换为$\sec, \tan$,均不为$1$则降到齐次转化为有理式。
  降次要么用立方和公式,要么用配方。
\end{note}

~


\begin{theorem}[$\frac{a\sin x + b\cos x}{c \sin x + d\cos x}$型计算技巧]
  $\frac{a\sin x + b\cos x}{c \sin x + d\cos x}$型
  的积分把分子写为$A$倍分母加上$B$倍分母导数的形式,用待定系数法进行计算
\end{theorem}

\begin{corollary}[$\tan x, \cot x$分式型]
  类似于$\int \frac{a\mathrm{d} x}{b + c\tan x}, \int \frac{a \mathrm{d} x}{b + c \cot x}$都可以转换回
  $\frac{a \sin x + b \cos x }{c \sin x + d \cos x}$型积分进行计算
\end{corollary}

~

\begin{exercise}[$\frac{a\sin x + b\cos x}{c \sin x + d\cos x}$型计算]
  (1)重点:$\int \frac{\sin x}{\sin x + \cos x}\mathrm{d} x$
  (2)重点:$\int \frac{\mathrm{d} x}{3 + 5\tan x}$
\end{exercise}

\begin{solution}
  (1)设$\sin x = A(\sin x + \cos x )+ B(\cos x - \sin x)$,得到$A = \frac{1}{2}, B = - \frac{1}{2}$,
  因此
  \begin{equation*}
    I = \int \frac{\frac{1}{2}(\sin x + \cos x) - \frac{1}{2} \mathrm{d}(\sin x + \cos x)}{\sin x + \cos x}\mathrm{d} x = \frac{1}{2}x - \frac{1}{2}\ln \left| \sin x + \cos x \right| + c
  \end{equation*}

  (2)由于$\frac{1}{3 + 5\tan x} = \frac{\cos x}{5 \sin x + 3 \cos x}$,
  设$\cos x = A(5 \sin x + 3 \cos x) + B(5 \cos x - 3 \sin x)$,
  则$5A = 3B, 3A + 5B = 1$,得到$A = \frac{3}{34}, B = \frac{5}{34}$,
  得到
  \begin{equation*}
    I = \frac{3}{34} x + \frac{5}{34} \ln \left| 5 \sin x + 3 \cos x \right| + c
  \end{equation*}
\end{solution}

~

\begin{theorem}[$\frac{1}{a \sin x + b \cos x + c}$积分]
  类似于$\int \frac{\mathrm{d} x}{a \sin x + b \cos x + c}$的积分采用二倍角公式转换为二次,
  得到
  \begin{equation*}
    \int \frac{1}{2a \sin \frac{x}{2} \cos \frac{x}{2} + b(2 \cos^2 \frac{x}{2} - 1) + c} \mathrm{d} x
  \end{equation*}
  使用前面的积分方法,上下同时除以$\cos^2 \frac{x}{2}$,
  再根据$\sec^2 x = \tan^2 x + 1$进行配方即可。
\end{theorem}


~

\begin{exercise}[相关训练]
  (1)$\int \frac{\mathrm{d} x}{\sin x + 2 \cos x + 3}$
\end{exercise}

\begin{solution}
  (1)$\sin x + 2 \cos x + 3 = 1 + 2 \sin \frac{x}{2} \cos \frac{x}{2} + 4 \cos^2 \frac{x}{2}$,
  因此
  \begin{equation*}
    I = \int \frac{\sec^2 \frac{x}{2}}{4 + 2 \tan \frac{x}{2} + \sec^2 x} = 2 \int \frac{d(\tan \frac{x}{2} + 1)}{(\tan \frac{x}{2} + 1)^2 + 4} = \arctan \frac{\tan \frac{x}{2} + 1}{2} + c
  \end{equation*}
\end{solution}

~

\begin{corollary}[$\frac{p \sin x + q \cos x + s}{a \sin x + b \cos x + c}\mathrm{d} x$]
  这类积分可以将分子写为$A$倍分母加$B$倍分母导数加$C$常数,
  其中常数项用前一个定理进行计算。
\end{corollary}

~

\begin{exercise}[三角根号]
  (1)$\int \sin x \cos x \sqrt{a^2 \sin^2 x + b^2 \cos ^2 x}\mathrm{d} x$
\end{exercise}

\begin{solution}
  (1)写为$\frac{1}{2}\int \sqrt{(a^2 - b^2)\sin^2 x + b^2}\mathrm{d}(\sin^2 x)$即可
\end{solution}

\subsection{对数与指数}

\begin{exercise}[几道经典指数积分]
  (1)$\int \frac{\mathrm{d}x}{e^x + e^{-x}}$
  (2)$\int \frac{\mathrm{d} x}{\sqrt{1 + e^{2x}}}$
  (3)$\int e^x \left( \frac{1-x}{1+x^2} \right)^2 \mathrm{d} x$
\end{exercise}

\begin{solution}
  (1)本质是$e^x$的有理函数,故$I = \int \frac{\mathrm{d}(e^x)}{e^{2x} + 1} = \arctan (e^x) + c$

  (2)同乘$e^{-x}$,故$I = -\int \frac{\mathrm{d}(e^{-x})}{\sqrt{e^{-2x} + 1}} = - \ln |e^{-x} + \sqrt{e^{-2x} + 1}| + c$

  (3)$I = \int \frac{e^x}{1+x^2}\mathrm{d} x - \int \frac{e^x}{(1 + x^2)^2}\mathrm{d} (x^2 + 1)$,
  第一项实在是没法处理,
  只能期待第二项,第二项等于$\int e^x \mathrm{d}(\frac{1}{1+x^2})$,
  因此用分部积分得到
  \begin{equation*}
    \int \frac{e^x}{1 + x^2}\mathrm{d} x + \int e^x \mathrm{d}(\frac{1}{1+x^2}) = \frac{e^x}{1 + x^2} + c
  \end{equation*}
\end{solution}

~

\begin{exercise}[几道经典对数积分]
  (1)$\int x^n \ln x \mathrm{d} x$
  (2)$\int \left( \frac{\ln x}{x} \right)^2 \mathrm{d} x$
  (3)$\int \sqrt{x} \ln^2 x \mathrm{d} x$
  (4)特殊:$\int \frac{1}{1-x^2} \ln \frac{1+x}{1-x}\mathrm{d} x$
\end{exercise}

\begin{solution}
  (1)列表法得到:$I = \frac{x^{n+1}\ln x}{n+1} - \int \frac{x^n}{n+1}\mathrm{d} x = \frac{x^{n+1}}{1+n}(\ln x - \frac{1}{n+1})+ c$
  \begin{center}
    \begin{tabular}[htp]{|c|c|}
      \hline
      $\ln x$& $\frac{1}{x}$\\
      \hline
      $x^n$&$\frac{1}{n+1} x^{n+1}$\\
      \hline
    \end{tabular}
  \end{center}

  (2)无法直接列表,因为中间项会交叉,因此可以用表格法一步一步列,
  \begin{equation*}
    I = - \frac{\ln^2 x}{x} + \int \frac{2\ln x}{x^2}\mathrm{d} x = - \frac{\ln^2 x}{x} - \frac{2\ln x}{x} + \int \frac{2}{x^2} \mathrm{d} x = - \frac{1}{x} (\ln^2 x + 2 \ln x + 2)+c
  \end{equation*}

  (3)同理用表格法一步一步列,
  \begin{equation*}
    I = \frac{2}{3}x^{\frac{3}{2}}\ln^2 x  - \frac{4}{3} \int x^{\frac{1}{2}} \ln x \mathrm{d} x = \frac{2}{3}x^{\frac{3}{2}}\ln^2 x  - \frac{4}{3} \left( \frac{2}{3} \ln x x^{\frac{3}{2}} - \frac{2}{3}\int x^{\frac{1}{2}} \mathrm{d} x\right)
  \end{equation*}

  (4)一看就是特殊配凑的,要注意到$\frac{1}{1-x^2}\mathrm{d} x = \frac{1}{2} \mathrm{d} \left( \ln \frac{1+x}{1-x} \right)$
\end{solution}

\subsection{分式问题}

\begin{exercise}[几个基础型]
  (1)$\int \frac{\mathrm{d}x}{ax^2 + b}(a > 0)$
  (2)重点:$\int \frac{\mathrm{d}x}{(ax^2 + b)^2}$
  (3)$\int \frac{\mathrm{d}x}{(ax^2 + b)^n}$

  (4)重点:$\int \frac{\mathrm{d}x}{ax^2 + bx + c}$
  (5)重点:$\int \frac{x}{ax^2 + bx + c}\mathrm{d}x$

  % (6)
  % $\int \frac{x^2}{ax^4 + bx^2 + c} dx$
\end{exercise}

\begin{solution}
  (1)原式$= \frac{1}{a} \int \frac{\mathrm{d}x}{x^2 + \frac{b}{a}}$,
  需要分类讨论:
  $b > 0$时变为$\frac{1}{a} \int \frac{\mathrm{d}x}{x^2 + (\sqrt{\frac{b}{a}})^2} = \frac{1}{a} \cdot \sqrt{\frac{a}{b}} \arctan (\sqrt{\frac{a}{b}}x)$。
  $b < 0$时变为$\frac{1}{a} \int \frac{\mathrm{d}x}{x^2 + (\sqrt{ - \frac{b}{a}})^2}$,后面同理

  (2)原式$= - \int \frac{1}{2ax} \mathrm{d} \frac{1}{ax^2 + b}
  = - \frac{1}{2ax} \cdot \frac{1}{ax^2 + b} + \int \frac{1}{ax^2 + b} \mathrm{d} \frac{1}{2ax}
  = - \frac{1}{2ax} \cdot \frac{1}{ax^2 + b} - \int \frac{1}{ax^2 + b} \cdot \frac{1}{2ax^2}\mathrm{d}x$,
  再对$\frac{1}{2ax^2 (ax^2 + b)} = \frac{\frac{1}{b}}{2ax^2} + \frac{- \frac{1}{2b}}{ax^2 + b}$有理分解,
  代入后全变为基础型1。

  (3)
  同理转化为$\int \frac{1}{2ax(1 - n)} \mathrm{d}(\frac{1}{(ax^2 + b)^{n-1}})$,
  再用分部积分去做。

  (4)
  原式$= \frac{1}{a} \int \frac{\mathrm{d}x}{x^2 + \frac{b}{a} x + \frac{c}{a}} =
  \frac{1}{a} \int \frac{\mathrm{d}x}{(x + \frac{b}{2a})^2+ \frac{c}{a} - \frac{b^2}{4a^2}} $,
  将$\frac{c}{a} - \frac{b^2}{4a^2}$看作整体,变为(1)

  (5)原式$= \frac{1}{2a} \int \frac{2ax}{ax^2 + bx + c}
  = \frac{1}{2a} \int \frac{2ax + b}{ax^2 + bx + c} - \frac{b}{ax^2 + bx + c}\mathrm{d}x
  = \frac{1}{2a}(\int \frac{\mathrm{d}(ax^2 + bx + c)}{ax^2 + bx + c} - \int \frac{b}{ax^2 + bx + c}\mathrm{d}x)$,
  变为(4).

  % (6)
  % 一般这种形式令$x = \frac{1}{t}$,
  % 可以把分子消去,化为技巧型2.
\end{solution}

~

\begin{exercise}[次方和差]
  尽量记住下面的分解方法:

  (1)$\int \frac{\mathrm{d} x}{x^3 + 1}$
  (2)$\int \frac{\mathrm{d} x}{x^3 - 1}$
  
  (3)$\int \frac{\mathrm{d} x}{x^4 - 1}$
  (4)重点:$\int \frac{\mathrm{d} x}{x^4 + 1}$(一定要记住!)

  (5)$\int \frac{\mathrm{d} x}{x^6 - 1}$
  (6)$\int \frac{\mathrm{d} x}{x^6 + 1}$
\end{exercise}

\begin{solution}
  (1)写为$\frac{1}{x^3 + 1} = \frac{1}{3(x+1)} + \frac{-x + 2}{3(x^2 - x + 1)}$,后者给分母配方即可

  (2)用立方差即可

  (3)用$\frac{1}{x^4 - 1} = \frac{1}{2} \left( \frac{1}{x^2 - 1} - \frac{1}{x^2 + 1} \right)$

  (4)构造$M = \int \frac{x^2 - 1}{x^4 + 1}\mathrm{d} x, N =\int \frac{x^2 + 1}{x^4 + 1}$,得到
  \begin{equation*}
    \begin{cases}
      M = \int \frac{1 - \frac{1}{x^2}}{x^2 + \frac{1}{x^2}} =  \int \frac{\mathrm{d}(x + \frac{1}{x})}{(x + \frac{1}{x})^2 - 2}\\ 
      N = \int \frac{x^2 + 1}{x^4 + 1} \mathrm{d} x= \int \frac{1 + \frac{1}{x^2}}{x^2 + \frac{1}{x^2}}\mathrm{d} x = \int \frac{\mathrm{d}(x - \frac{1}{x})}{(x-\frac{1}{x})^2 + 2}
    \end{cases}
  \end{equation*}
  因此所求即$\frac{1}{2}(N - M)$

  (5)用平方差(不要用立方差):$\frac{1}{x^6 - 1} = \frac{1}{2}(\frac{1}{x^3 - 1} + \frac{1}{x^3 + 1})$,
  转换为(1)(2)

  (6)类似四次方的讨论+立方和
  \begin{align*}
    I &= \frac{1}{2}\int \frac{x^4 + 1}{x^6 + 1}\mathrm{d} x - \frac{1}{2} \int \frac{x^4 - 1}{x^6 + 1}\mathrm{d} x\\
    &= \frac{1}{2} \int \frac{x^2 + (x^4 - x^2 + 1)}{x^6 + 1}\mathrm{d} x - \frac{1}{2} \int \frac{(x^2 - 1)(x^2 + 1)}{(x^2 + 1)(x^4 - x^2 + 1)}\mathrm{d} x\\
    &= \frac{1}{2} \int \frac{x^2}{x^6 + 1}\mathrm{d} x + \frac{1}{2}\int \frac{\mathrm{d} x}{x^2 + 1} - \frac{1}{2} \int \frac{x^2 - 1}{x^{4 }- x^2 + 1}\mathrm{d} x\\
    &= \frac{1}{6}\int \frac{\mathrm{d}(x^3)}{(x^3)^2 + 1} + \frac{1}{2} \arctan x - \frac{1}{2} \int \frac{\mathrm{d}(x + \frac{1}{x})}{(x + \frac{1}{x})^2 + 3}\mathrm{d} x
  \end{align*}
\end{solution}

~

\begin{exercise}[几个经典题]
  (1)$\int \frac{x^2}{(1 - x)^{100}}\mathrm{d} x$
  (2)$\int \frac{\mathrm{d} x}{(x+a)^2 (x + b)^2}$
\end{exercise}

\begin{solution}
  (1)显然待定系数法有理分解不可行,因此用技巧$x^2 = [(1-x) - 1]^2$得到$\frac{x^2}{(1-x)^{100}} = \frac{(1-x)^2 - 2(1-x) + 1}{(1-x)^{100}}$,做换元即可

  (2)这种有理分解比较常见,最好记下来,
  \begin{equation*}
    \frac{1}{(x+a)^2(x+b)^2} = \frac{1}{(b-a)^2} \left( \frac{1}{x+a} - \frac{1}{x+b} \right)^2 = \frac{1}{(b-a)^2} \left[ \frac{1}{(x+a)^2} + \frac{1}{(x+b)^2} - \frac{2}{(x+a)(x+b)} \right]
  \end{equation*}
\end{solution}

\subsection{根式问题}

\begin{exercise}[三个基础公式的证明]
  (1)$\int \sqrt{a^2 - x^2}\mathrm{d} x$
  (2)重点:$\int \sqrt{x^2 + a^2}\mathrm{d} x$
  (3)重点:$\int \sqrt{x^2 - a^2}\mathrm{d} x$
\end{exercise}

\begin{solution}
  (1)设$x = a \sin t$,
  则$I = \int a \cos t \cdot a\cos t\mathrm{d} t = \frac{a^2}{2} \int (1 + \cos 2t)\mathrm{d} t = \frac{a^2}{2}(t + \sin t \cos t)+ C$。
  画一个三角形做辅助,角$t$对着$x$,斜边为$a$,
  因此$\sin t = \frac{x}{a}, \cos t = \frac{\sqrt{a^2 - x^2}}{a}$,
  带回得到:
  \begin{equation*}
    I = \frac{1}{2}(x \sqrt{a^2 - x^2} + a^2 \arcsin \frac{x}{a}) + C
  \end{equation*}

  (2)$x = a \tan t$,$I = \int a^2 \sec^3 x \mathrm{d} t =  \frac{a^2}{2} (\sec x \tan x ) + \frac{a^2}{2} \ln |\tan x + \sec x|$(一定要记住$\sec x, \sec^3 x$的积分,如果一次方都没记住,就死翘翘了。。),
  因此得到结果为
  \begin{equation*}
    \frac{1}{2} (x \sqrt{x^2 + a^2} + a^2 \ln |x + \sqrt{x^2 + a^2}|) + C
  \end{equation*}

  (3)设$x = a \sec t$,得到$I = a^2\int \tan^2 t \sec t \mathrm{d}t = a^2 \int \sec^3 t - \sec t \mathrm{d} t$,
  背公式以及画三角形得到:
  \begin{equation*}
    I = \frac{a^2}{2}(\sec t \tan t - \ln |\tan t + \sec t|) + C = \frac{1}{2}(x \sqrt{x^2 - a^2} - a^2 \ln |x + \sqrt{x^2 - a^2}|) + C
  \end{equation*}
\end{solution}

~

\begin{exercise}[$(x-a)(b-x)$型]
  (1)$\int \frac{\mathrm{d} x}{\sqrt{x(1-x)}}$
  (2)$\int \frac{\mathrm{d} x}{\sqrt{(x -a)(b - x)}}$
  (3)$\int \sqrt{(x-a)(b-x)}\mathrm{d} x$
\end{exercise}

\begin{solution}
  (1)设$x = \sin^2 t$,则$I = \int \frac{2\sin t \cos t}{\sin t \cos t}\mathrm{d} t = 2t + C$,
  直接带回得到$I = 2 \arcsin \sqrt{x} + C$

  (2)令$t = x-a$,则变为$\int \frac{\mathrm{d} t}{\sqrt{t(b - a - t)}}$,
  令$t = (b - a) \sin^2 s$即可,
  故
  \begin{equation*}
    I = \int \frac{2(b-a)\sin s \cos s \mathrm{d} s}{(b-a) \sin s \cos s} = 2s = 2 \arcsin \sqrt{\frac{t}{(b-a)}} = 2 \arcsin  \sqrt{\frac{x-a}{b-a}}+C
  \end{equation*}

  (3)同理得到$I = \int 2(b-a)^2 \sin^2 s \cos^2 s\mathrm{d} s = \frac{(b-a)^2}{4} \int (1 - \cos 4t )\mathrm{d} t$,
  后续略
\end{solution}

~

\begin{exercise}[其他特别经典的根式积分问题]
  (1)$\int \left( \sqrt{\frac{1+x}{1-x}} + \sqrt{\frac{1-x}{1+x}} \right)\mathrm{d}x $

  (2)$\int \frac{\mathrm{d} x}{(\sqrt{x} + 1)(x + 3)}$

  (3)$\int \frac{\mathrm{d} x}{x \sqrt{x^2 - 2x - 3}}$

  (4)$I = \frac{\mathrm{d} x}{x + \sqrt{x^2 - x + 1}}$
\end{exercise}

\begin{solution}
  (1)让分母一样:$\frac{1+x}{\sqrt{1-x^2}} + \frac{1-x}{\sqrt{1-x^2}} = \frac{2}{\sqrt{1-x^2}}$,
  直接转换为了$\arcsin x$的积分

  (2)令$t = \sqrt{x}$,
  则$I = \int \frac{2t}{(t+1)(t^2 + 3)}\mathrm{d} t = \int \frac{A}{t+1}\mathrm{d} t + \int \frac{Bt + C}{t^2 + 3}\mathrm{d}t$,
  解出$A = - \frac{1}{2}, B = \frac{1}{2}, C = \frac{3}{2}$,
  从而
  \begin{equation*}
    I = - \frac{1}{2} \ln |t+1| + \int \frac{1}{4} \ln|t^2 + 3| + \frac{\sqrt{3}}{2} \arctan \frac{t}{\sqrt{3}} + C
  \end{equation*}

  (3.1)方法1:使用Euler变换。
  设$\sqrt{x^2 - 2x + 3} = x - t$,则得到$x = \frac{t^2 + 3}{2(t-1)}$,
  进而$\mathrm{d} x = \frac{t^2 - 2t + 3}{2(t-1)^2}\mathrm{d}t$,
  此时积分变为
  \begin{equation*}
    I = - \int \frac{2}{t^2 + 3} \mathrm{d} t  = - \frac{2}{\sqrt{3}} \arctan \frac{t}{\sqrt{3}} + C
  \end{equation*}

  (3.2)方法2:配方+三角函数(一般还是推荐三角函数)。
  $I = \int \frac{\mathrm{d} x}{x \left[ (x-1)^2 - 4 \right]^{\frac{1}{2}}}$,
  做换元$x - 1 = 2 \sec t$,
  因此
  \begin{equation*}
    I = \int \frac{2 \sec t \tan t}{(2 \sec t + 1) 2 \tan t}\mathrm{d} t = \int \frac{\mathrm{d} t}{2 + \cos t}
  \end{equation*}
  背公式即$\frac{2}{\sqrt{1-a^2}} \arctan \left( \sqrt{\frac{1-a}{1+a}}\tan \frac{x}{2} \right) + C$,
  代入$a = \frac{1}{2}$。
  注意带回时需要用半角公式,$\tan \frac{t}{2} = \frac{\sin t}{1 + \cos t}$,再画个三角形进行辅助。

  (4)可以用Euler变换或者配方三角替换。
\end{solution}

\subsection{配对积分法}

\begin{exercise}[经典配对积分法]
  (1)计算$I_1 = \int e^{ax } \cos b x \mathrm{d} x, I_2 = \int e^{ax} \sin bx \mathrm{d} x$
  (2)计算$I_1 = \sin (\ln x)\mathrm{d} x , I_{2 }= \int \cos(\ln x)\mathrm{d} x$
\end{exercise}

\begin{solution}
  (1)用分部积分得到$I_1 = \frac{1}{a}(e^{ax} \cos bx + bI_2)$,
  同理$I_2 = \frac{1}{a}(e^{ax} \sin x - bI_1)$,
  列出线性方程组:
  \begin{equation*}
    \begin{cases}
      aI_1 - bI_2 = e^{ax} \cos bx\\
      bI_1 + aI_2 = e^{ax} \sin bx
    \end{cases}
    \Rightarrow
    \begin{cases}
      I_1 = \frac{e^{ax}}{a^2 + b^2} (a \cos bx + b \sin bx) + C\\
      I_2 = \frac{e^{ax}}{a^2 + b^2}(a \sin bx - b \cos bx) + C
    \end{cases}
  \end{equation*}

  (2)直接用分部积分:$\int \sin(\ln x)\mathrm{d} x = x \sin(\ln x) - \int \cos(\ln x)\mathrm{d} x$,
  同理$\int \cos(\ln x)\mathrm{d} x = x \cos(\ln x) + \int \sin(\ln x)\mathrm{d} x$
\end{solution}

\subsection{递推问题}

\begin{exercise}[几个最常见递推]
  (1)$I_n = \int \frac{\sin nx}{\sin x}\mathrm{d} x$
  (2)$I_n = \int \sec^n x\mathrm{d} x$
  (3)$I_n = \int e^{ax} \sin ^n x\mathrm{d} x$
\end{exercise}

\begin{solution}
  (1)
  首先根据和差公式和积化和差:
  \begin{equation*}
    \sin nx = \sin[(n-1)x + x] = \sin (n-1)x \cos x + \cos(n-1) x \sin x = \frac{1}{2}[\sin nx + \sin (n-2)x] + \cos (n-1)x \sin x
  \end{equation*}
  因此
  \begin{equation*}
    I_n = \frac{1}{2}(I_n + I_{n-2}) + \frac{1}{n-1} \sin(n-1)x \Rightarrow I_n = \frac{2}{n-1}\sin(n-1)x + I_{n-2}
  \end{equation*}

  (2)先凑出$\mathrm{d}\tan x$并使用分部积分:
  \begin{equation*}
    \int \sec^n x\mathrm{d} x = \int \sec^{n-2}x \mathrm{d}(\tan x) = \tan x \sec^{n-2}x - (n-2) \int \tan^2 x \sec^{n-2}x \mathrm{d} x
  \end{equation*}
  再代入$\tan^2 x = \sec^2 x - 1$:
  \begin{equation*}
    I_n = \tan x \sec^{n-2} x - (n-2)(I_n - I_{n-2}) \Rightarrow I_n = \frac{\tan x \sec^{n-2}x}{n - 1}+\frac{n-2}{n-1} I_{n-2}
  \end{equation*}

  (3)
\end{solution}

\begin{note}
  $\int \sec x \mathrm{d} x = \ln |\tan x + \sec x|$和
  $\int \sec^3 x\mathrm{d} x = \frac{1}{2} \tan x \sec x + \frac{1}{2} \ln|\tan x + \sec x|$在根式中经常用到,要记住,具体证明过程见前面部分。
\end{note}