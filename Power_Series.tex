




\chapter{幂级数}

\section{幂级数概念及其收敛区间}

\begin{definition}[幂级数]
  $\sum\limits_{n = 0}^{\infty}a_n(x - x_0)^n$称为幂级数,
  但一般只研究$\sum\limits_{n = 0}^{\infty}a_nx^n$
\end{definition}

\subsection{收敛区间与收敛半径}

\begin{theorem}[Abel定理]
  若幂级数$\sum\limits_{n = 1}^{\infty}a_nx^n$在$x_0 \neq 0$收敛,
  则对$\forall x, |x| < |x_0|$,
  $\sum\limits_{n = 1}^{\infty}a_nx^n$绝对收敛;
  若$x_0$处发散,则$\forall x, |x| > |x_0|$均发散。
\end{theorem}

\begin{proof}
  若$\sum\limits_{n = 0}^{\infty}a_nx_0^n$收敛,
  则$|a_nx^n| = |a_nx_0^n| \cdot |\frac{x}{x_0}|^n \leq Mr^n$,
  这里$r = |\frac{x}{x_0}| \in [0,1)$。
  另一侧同理
\end{proof}


\begin{definition}[收敛半径]
  根据Abel定理可知$\exists R \in \mathbb{R}$,使得$|x| < R$时幂级数绝对收敛(若$R = 0$则仅在$x = 0$处收敛),
  对一切$|x| > R$均发散,
  称$R$为幂级数收敛半径,$(-R,R)$为收敛区间。
\end{definition}

\begin{theorem}[收敛半径的计算]
  幂级数收敛半径$R$(可以为$0, \infty$)的计算有以下几种方式:
  \begin{itemize}
  \item 根式(柯西-阿达马):$\varlimsup \limits _{n \rightarrow \infty}  \sqrt[n]{|a_n|} = \frac{1}{R}$(可以为$0,\infty$)
  \item 比式(一般用得少):若$\lim \limits _{n \rightarrow \infty} \frac{|a_{n+1}|}{|a_n|} = \frac{1}{R}$,也能推出$\lim \limits _{n \rightarrow \infty} \sqrt[n]{a_n} = \frac{1}{R}$
  \end{itemize}
\end{theorem}

\begin{proof}
  (1)根式:只需要考虑$\sum\limits_{n = 0}^{\infty}|a_nx^n|$,
  用数项级数的根式判别法为$\varlimsup \limits_{n \rightarrow \infty} \sqrt[n]{|a_nx^n|} = \varlimsup \limits_{n \rightarrow \infty} \sqrt[n]{|a_n|} \cdot |x|$,
  因此若记$\varlimsup \limits_{n \rightarrow \infty} \sqrt[n]{|a_n|} = \rho$,
  则$|x| < \frac{1}{\rho}$则收敛,$\frac{1}{\rho}$为收敛半径。
\end{proof}

~

\begin{exercise}[计算收敛半径]
  计算收敛半径与收敛域
  (1)$\sum\limits_{n = 1}^{\infty}\frac{x^n}{n^2}$
  (2)$\sum\limits_{n = 1}^{\infty}\frac{x^n}{n}$
  (3)$\sum\limits_{n = 1}^{\infty}\frac{(2n-1)!!}{(2n)!!}x^n$
\end{exercise}

\begin{solution}
  (1)$\lim \limits _{n \rightarrow \infty} \sqrt[n]{\frac{1}{n^2}} = 1$,
  收敛半径$1$,收敛域$[-1,1]$

  (2)$\lim \limits _{n \rightarrow \infty} \sqrt[n]{\frac{1}{n}} = 1$,
  收敛半径$1$,收敛域$[-1,1)$

  (3)$a_n = \frac{(2n-1)!!}{(2n)!!}$,
  用Wallis+根式或者比式$\lim \limits _{n \rightarrow \infty} \frac{a_{n+1}}{a_n} = \lim \limits _{n \rightarrow \infty} \frac{2n+1}{2n+2} = 1$得到收敛半径为$1$。
  而$\frac{(2n-1)!!}{(2n)!!} \sim \frac{1}{\sqrt{n\pi}}$,因此发散。
  $\sum\limits_{n = 1}^{\infty}\frac{(2n-1)!!}{(2n)!!}(-1)^n$根据Leibniz可知收敛。
  故收敛域为$[-1,1)$
\end{solution}

\begin{note}
  上述题目也说明在逐项求导后,端点处收敛域可能发生变换
\end{note}

~

\begin{exercise}[常规格式幂级数]
  (1)求收敛域:$\sum\limits_{n = 1}^{\infty}(1 + \frac{1}{2} + \cdots + \frac{1}{n})x^n$
  $\sum\limits_{n = 1}^{\infty}\frac{x^n}{1 + \frac{1}{2} + \cdots + \frac{1}{n}}$

  (2)$0 < a < b$,求$\sum\limits_{n = 1}^{\infty}\frac{x^n}{a^n + b^n}, \sum\limits_{n = 1}^{\infty}(a^n + b^n)x^n$收敛域

  (3)$\sum\limits_{n = 1}^{\infty}\frac{x^{2n}}{n - 3^{2n}}, \sum\limits_{n = 1}^{\infty}\frac{x^{n^2}}{2^n}$收敛域

  (4)$\sum\limits_{n = 1}^{\infty}\frac{3^n + (-2)^n}{n}(x + 1)^n$
\end{exercise}

\begin{solution}
  (1)要么根据Euler常数得到$\sqrt[n]{1 + \frac{1}{2} + \cdots + \frac{1}{n}} \sim \sqrt[n]{\ln n} = 1$,
  要么用$1 \leq 1 + \frac{1}{2} + \cdots + \frac{1}{n} \leq n$得到收敛半径为$R = 1$,
  显然收敛域为$(-1,1)$
  第二个同理得到收敛域为$[-1,1]$

  (2)第一个$\lim \limits _{n \rightarrow \infty} \sqrt[n]{\frac{1}{a^n + b^n}} = \frac{1}{b}$,
  收敛半径为$b$,$x = \pm b$都震荡发散。
  第二个$\lim \limits _{n \rightarrow \infty} \sqrt[n]{a^n + b^n} = b$,收敛半径$b$,
  $x = \pm b$震荡发散

  (3)由于不是$x^n$因此开$2n$次根号,
  $\lim \limits _{n \rightarrow \infty} \sqrt[2n]{|\frac{1}{n - 3^{2n}}|} = \frac{1}{3}$,
  半径为$3$,$x = \pm 3$时通项不趋于$0$,因此发散。
  $\lim \limits _{n \rightarrow \infty} \sqrt[n^2]{\frac{1}{2^n}} = 1$,
  而$x = \pm 1$均收敛

  (4)将$x+1$视为$y$即可,
  $\lim \limits _{n \rightarrow \infty} \sqrt[n]{\frac{3^n + (-2)^n}{n}} = 3$,
  收敛半径为$\frac{1}{3}$,$y = \frac{1}{3}$时为$\sum\limits_{n = 1}^{\infty}(\frac{1}{n} + \frac{1}{n}(-\frac{2}{3})^n)$前者发散,后者收敛,因此整体发散。
  $y = -\frac{1}{3}$时为$\sum\limits_{n = 1}^{\infty}(\frac{(-1)^n}{n} + \frac{1}{n}(\frac{2}{3})^n)$收敛,
  因此收敛域为$[-\frac{1}{3} - 1, \frac{1}{3} - 1)$即$x \in [- \frac{4}{3}, - \frac{2}{3})$
\end{solution}

~

\begin{exercise}[非常规格式幂级数]
  (1)$\sum\limits_{n = 1}^{\infty}\frac{(x^2 + x + 1)^n}{3n}$的收敛域
  (2)$\sum\limits_{n = 1}^{\infty}\frac{1}{2^n}\left( \frac{x - 1}{2x + 1} \right)^{n^2}$
  (3)$\sum\limits_{n = 1}^{\infty}\frac{1^n + 2^n + \cdots + 2022^n}{n^{2022}}\left( \frac{1 - x}{1 + x} \right)^n$
\end{exercise}

\begin{solution}
  (1)设$y = x^2 + x + 1$,
  考虑$\lim \limits _{n \rightarrow \infty} \sqrt[n]{\frac{1}{3n}} = 1$,$R = 1$,
  而$y = 1$发散,$y = -1$收敛,
  计算$x^2 + x + 1 \in [-1,1)$,
  即$x^2 + x \in [-2,0)$,
  注意最终得到$x \in (-1,0)$两侧都是开的!

  (2)考虑$y = \frac{x - 1}{2x + 1}$,
  计算得到$y$的收敛域为$[-1,1]$,
  计算$-1 \leq \frac{x - 1}{2x + 1} \leq 1$,
  直接讨论分母容易出错,因此这里看成两个不等式。
  $\frac{x - 1}{2x + 1} + 1 = \frac{3x}{2x + 1} \geq 0$以及$\frac{x - 1}{2x + 1} - 1 = \frac{-x - 2}{2x + 1} \leq 0$,
  第一个要求$x \geq 0$或者$x < - \frac{1}{2}$,
  第二个要求$x \leq -2$或者$x \geq - \frac{1}{2}$,
  求交得到$x \in (-\infty,-2] \cup [0,+\infty)$

  (3)$y = \frac{1-x}{1+x}$得到收敛半径为$\frac{1}{2022}$且端点收敛,
  考虑$- \frac{1}{2022} \leq \frac{1-x}{1+x} \leq \frac{1}{2022}$,
  分为两部分,
  得到$\frac{2023 - 2021x}{1 + x} \geq 0$以及$\frac{2021 - 2023x}{1 + x} \leq 0$,
  前者解出$-1 < x \leq \frac{2023}{2021}$,后者解出$x < -1$或者$x \geq \frac{2021}{2023}$,
  得出结果为$x \in [\frac{2021}{2023}, \frac{2023}{2021}]$
\end{solution}

\subsection{幂级数的一致收敛性}

下面考虑幂级数$\sum\limits_{n = 1}^{\infty}a_nx^n$的一致收敛性,
假设其收敛半径为$R$,收敛域在端点处需要讨论,
下面开始分析。

\begin{theorem}[Abel第二定理]
  若幂级数收敛半径为$R$,
 $\sum\limits_{n = 1}^{\infty}a_nx^n$在$\langle - R, R \rangle $内闭一致收敛(端点未确定)
\end{theorem}

\begin{proof}
  设$\sum\limits_{n = 0}^{\infty}a_nx^n_0$收敛,
  (1)若$x_0 > 0$,则$\forall x \in [0,x_0]$有$\sum\limits_{n = 0}^{\infty}a_nx^n = \sum\limits_{n = 0}^{\infty}a_nx_0^n \left( \frac{x}{x_0} \right)^n$,根据Abel判别法可证。
  (2)若$x_0 < 0$,则$\forall x \in [x_0,0]$有$\sum\limits_{n = 0}^{\infty}a_nx^n = \sum\limits_{n = 0}^{\infty}a_nx_0^n \left( \frac{x}{x_0} \right)^n$,根据Abel可证。
  进而任选闭区间都一致收敛。
\end{proof}


\begin{theorem}[连续性]
  幂级数的和函数$\sum\limits_{n = 0}^{\infty}a_nx^n$在收敛域$\langle -R, R\rangle$是连续的。
\end{theorem}


\begin{theorem}[区间不变性]
  幂级数$\sum\limits_{n = 0}^{\infty}a_nx^n$的收敛半径为$R$,
  则其逐项求导和逐项积分后得到的幂级数的收敛半径仍为$R$,
  不过端点处的一致收敛情况可能改变!
\end{theorem}

\begin{proof}
  考虑$\sum\limits_{n = 0}^{\infty}a_nx^n$的逐项求导与逐项积分,
  得到$\sum\limits_{n = 1}^{\infty}na_n$和$\sum\limits_{n = 0}^{\infty}\frac{a_n}{n+1}x^{n+1}$,
  根据$\varlimsup \limits _{n \rightarrow \infty} \sqrt[n]{|a_n|} = \varlimsup \limits_{n \rightarrow \infty}\sqrt[n-1]{|na_n|} = \varlimsup \limits_{n \rightarrow \infty}\sqrt[n-1]{|\frac{a_n}{n+1}|}$可知收敛半径相同
\end{proof}

\begin{note}
  端点处一致收敛性不保证,例如$\sum\limits_{n = 1}^{\infty}\frac{x^n}{n}$收敛域为$[-1,1)$,
  而$\sum\limits_{n = 1}^{\infty}x^{n-1}$收敛域为$(-1,1)$
\end{note}

\begin{theorem}[求导与求积]
  设$(-R,R)$上$f(x) = \sum\limits_{n = 0}^{\infty}a_nx^n$,则对$\forall x \in (-R, R)$有:
  \begin{equation*}
    f^{\prime}(x) = \sum\limits_{n = 1}^{\infty}na_nx^{n-1}, \int_0^x f(t) \mathrm{d} t = \sum\limits_{n = 0}^{\infty}\frac{a_n}{n+1}x^{n+1}
  \end{equation*}
\end{theorem}

\begin{note}
  幂级数的逐项求导和逐项积分要注意其只能保证在收敛区间成立,但不保证在收敛域上成立!
\end{note}

\subsection{缺项幂级数的收敛区域}

\begin{exercise}[缺项幂级数的收敛范围]
  (1)求$\sum\limits_{n = 1}^{\infty}n^{n^2}x^{n^3}$的收敛范围
\end{exercise}

\begin{solution}
  令$a_k =
  \begin{cases}
    n^{n^2}, & k = n^3\\
    0, & k \neq n^3
  \end{cases}
  $,
  则$\sum\limits_{n = 1}^{\infty}n^{n^2}x^{n^3} = \sum\limits_{k = 1}^{\infty}a_kx^k$,
  此时
  \begin{equation*}
    \frac{1}{R} =\varlimsup \limits_{k \rightarrow \infty}\sqrt[k]{|a_k|} =  \lim \limits _{n \rightarrow \infty} \sqrt[n^3]{n^{n^2}} = \lim \limits _{n \rightarrow \infty} (n^{n^2})^{\frac{1}{n^3}} = 1
  \end{equation*}
\end{solution}



\section{幂级数展开式求和}


\begin{theorem}[必备幂级数展开]
  以下的幂级数展开会经常使用,基本上是分母较为简单的Taylor展开式
  \begin{enumerate}
  \item $\frac{1}{1-x} = \sum\limits_{n = 0}^{\infty} x^n$
  \item $e^x = \sum\limits_{n = 0}^{\infty} \frac{x^n}{n!}$
  \item $\ln (1 + x) = \sum\limits_{n = 1}^{\infty} (-1)^{n-1} \frac{x^n}{n}$
  \item $\arctan x = \sum\limits_{n = 0}^{\infty} \frac{(-1)^n x^{2n+1}}{2n+1}$
  \end{enumerate}
\end{theorem}

\begin{note}
  具体用哪个主要看$(-1)$的指数次和分母的关系。
\end{note}

\subsection{等比级数}

\begin{theorem}[等比级数]
  根据$f(x) = \sum\limits_{n = 0}^{\infty} x^n = \frac{1}{1-x}$可以推出:
  \begin{enumerate}
  \item $\sum\limits_{n = 1}^{\infty} nx^n = \frac{x}{(1 - x)^2}$,方法为对$\sum\limits_{n = 0}^{\infty}x^n$求导,再同乘$x$
  \item $\sum\limits_{n = 1}^{\infty} n^2x^n = \frac{x(1+x)}{(1-x)^3}$,方法为对$\sum\limits_{n = 1}^{\infty}nx^n$求导,再乘$x$
  \end{enumerate}
\end{theorem}


\begin{exercise}[等比级数]
  (1)$\sum\limits_{n = 1}^{\infty} \frac{n^2 + 3n}{2^n}$
  (2)$\sum\limits_{n = 1}^{\infty} \frac{x^n}{n(n+1)}$
\end{exercise}

\begin{solution}
  (1)这里展示正推,进行公式推导:
  \begin{equation*}
    \begin{cases}
      f_1(x) = \sum\limits_{n = 0}^{\infty} x^n = \frac{1}{1-x} \Rightarrow f^{\prime}_1(x) = \sum\limits_{n = 1}^{\infty} nx^{n-1} = \frac{1}{(1-x)^2}\\
      f_2(x) = \sum\limits_{n = 1}^{\infty} nx^n = \frac{x}{(1-x)^2} \Rightarrow f_2^{\prime}(x) = \sum\limits_{n = 1}^{\infty} n^2x^{n-1} = \frac{1+x}{(1 - x)^3}\\
      f_3(x) = \sum\limits_{n = 1}^{\infty} n^2x^n = \frac{x(1 + x)}{(1 - x)^3}
    \end{cases}
  \end{equation*}
  因此$I = f_3(\frac{1}{2}) + 3 f_2(\frac{1}{2}) = 6 + 6 = 12$

  (2)这里展示反推:
  \begin{equation*}
    \begin{cases}
      f_1(x) = \sum\limits_{n = 1}^{\infty} \frac{x^n}{n(n+1)}\\
      f_2(x) = \sum\limits_{n = 1}^{\infty} \frac{x^{n+1}}{n(n+1)} \Rightarrow f^{\prime}_2(x) = \sum\limits_{n = 1}^{\infty} \frac{x^n}{n} \Rightarrow f_2^{\prime\prime}(x) = \sum\limits_{n = 1}^{\infty}x^{n-1} = \sum\limits_{n = 0}^{\infty} x^n = \frac{1}{1-x}
    \end{cases}
  \end{equation*}
  因此推出$f_2^{\prime}(x) = - \ln(1 - x), f_2(x) = (1-x)\ln(1-x) + x$,
  因此$f_1(x) = \frac{1-x}{x}\ln(1-x) + 1$
\end{solution}


\subsection{$\arctan x$级数}

\begin{exercise}[$\arctan x$]
  (1)重点:求$\sum\limits_{n = 1}^{\infty} \frac{(-1)^{n-1}}{n(2n+1)}$
\end{exercise}

\begin{solution}
  (1)相差较大,选择裂项:
  $\sum\limits_{n = 1}^{\infty}\frac{(-1)^{n-1}}{n(2n+1)} = 2 \sum\limits_{n = 1}^{\infty} \left( \frac{(-1)^{n-1}}{2n} - \frac{(-1)^{n-1}}{2n+1} \right)$,
  两者均收敛,且
  \begin{equation*}
    \begin{cases}
      \ln(1 + x) = \sum\limits_{n = 1}^{\infty} \frac{(-1)^{n-1}x^n}{n}\\
      \arctan x = \sum\limits_{n = 0}^{\infty} \frac{(-1)^nx^{2n+1}}{2n+1}
    \end{cases}
  \end{equation*}
  因此结果为$\ln 2 + \frac{\pi}{2} - 2$
\end{solution}

\subsection{$e^x$级数}

\begin{exercise}[$e^x$]
  (1)计算$\lim \limits _{n \rightarrow \infty} \left[ 1 + \frac{1}{2!} + \frac{1}{4!} + \cdots + \frac{1}{(2n)!} \right]$
\end{exercise}

\begin{solution}
  (1)由于$e^x = \sum\limits_{n = 0}^{\infty}\frac{x^n}{n!}$,
  因此$e^{-x} = \sum\limits_{n = 0}^{\infty} \frac{(-1)^n x^n}{n!}$,
  故
  \begin{equation*}
    \frac{e^x + e^{-x}}{2} = \sum\limits_{n = 0}^{\infty}\frac{x^{2n}}{(2n)!}
  \end{equation*}
  取$x = 1$,得到$\sum\limits_{n = 0}^{\infty} \frac{1}{(2n)!} = \frac{e + e^{-1}}{2}$
\end{solution}

% \subsection{简单的幂级数求和:等比数列积分与求导}

% \begin{theorem}[求导基本结论]
%   常用$\sum\limits_{n = 1}^{\infty}n^ax^n$型幂级数:
%   (1)$\sum\limits_{n = 0}^{\infty}x^n = \frac{1}{1-x}$
%   (2)$\sum\limits_{n = 1}^{\infty}nx^n = \frac{x}{(1 - x)^2}$
%   (3)$\sum\limits_{n = 1}^{\infty}n^2x^n = \frac{x(1+x)}{(1 - x)^3}$
% \end{theorem}

% \begin{proof}
%   如果可以直接看出的话,可以不要根据题目反推,而是根据已知结论推导,更方便。还有尽量凑下标去分解。

%   (2)$\sum\limits_{n = 1}^{\infty}nx^n$,不是积分,
%   而是用$\sum\limits_{n = 0}^{\infty}x^n = \frac{1}{1-x}$求导得到$\sum\limits_{n = 1}^{\infty}nx^{n-1} = \frac{1}{(1-x)^2}$,
%   再乘上$x$,即$\frac{x}{(1-x)^2}$

%   (3)尽量凑分解,而不是去凑$x$次数:
%   $\sum\limits_{n = 1}^{\infty}n^2x^n  = \sum\limits_{n = 1}^{\infty}(n+1)(n+2)x^n - \sum\limits_{n = 1}^{\infty}(3n+2)x^n$,
%   因此结果为$\frac{x(1+x)}{(1-x)^3}$
% \end{proof}

% ~

% \begin{exercise}[$\sum n^ax^n$型]
%   (1)$\sum\limits_{n = 0}^{\infty}(3n + 5)x^n$
%   (2)$\sum\limits_{n = 1}^{\infty} n x^{3n-1}$
%   (3)$\sum\limits_{n = 1}^{\infty} \frac{2n+1}{2^n}x^{2n+1}$
%   (4)$\sum\limits_{n = 1}^{\infty}n(n+2)x^n$
%   (5)$\sum\limits_{n = 1}^{\infty}(-1)^{n-1}n^2x^n$
% \end{exercise}

% \begin{solution}
%   (1)$3 \sum\limits_{n = 0}^{\infty} nx^n + 5 \sum\limits_{n = 0}^{\infty}x^n = \frac{3x}{(1-x)^2} + \frac{5}{1-x}$

%   (2)即$\frac{1}{x}\sum\limits_{n = 1}^{\infty} n x^{3n} = \frac{1}{x} \cdot \frac{x^{3n}}{(1 - x^{3n})^2}$

%   (3)即$x \sum\limits_{n = 1}^{\infty}(2n+1)(\frac{x^2}{2})^n = 2x \sum\limits_{n = 1}^{\infty} n \left( \frac{x^2}{2} \right)^n + x \sum\limits_{n = 1}^{\infty} \left( \frac{x^2}{2} \right)^n$,
%   前者用结论,后者用等比数列求和即可,
%   结果为$\frac{6x^3 - x^5}{(2 - x^2)^2}$

%   (4)分解为$\sum\limits_{n = 1}^{\infty}n^2x^n + \sum\limits_{n = 1}^{\infty}2n x^n$,
%   结果为$\frac{3x - x^2}{(1-x)^3}$

%   (5)即$- \sum\limits_{n = 1}^{\infty}n^2(-x)^n = - \frac{(-x)(1 - x)}{(1 + x)^3} = \frac{x(1-x)}{(1+x)^3}$
% \end{solution}

% ~

% \begin{theorem}[积分基本结论]
%   (1)$\sum\limits_{n = 1}^{\infty}\frac{x^n}{n} = - \ln(1-x), x \in [-1,1)$(注意负号!)
%   (2)$\sum\limits_{n = 1}^{\infty} \frac{x^{n+1}}{n(n+1)} = x + (1-x)\ln (1 - x), x \in [-1,1]$
% \end{theorem}

% \begin{proof}
%   (1)直接用$\sum\limits_{n = 0}^{\infty}x^n = \frac{1}{1-x}$两边积分就行

%   (2)$\sum\limits_{n = 1}^{\infty} \frac{x^{n+1}}{n(n+1)}$用$\sum\limits_{n = 1}^{\infty}\frac{x^n}{n}$两边积分,
%   得到$\sum\limits_{n = 1}^{\infty} \frac{x^{n+1}}{n(n+1)} = x \ln(1 - x) + \int \frac{x}{1-x}\mathrm{d} x$,
%   而$\int \frac{x}{1-x}\mathrm{d} x = \int \frac{x - 1}{1 - x}\mathrm{d} x + \int \frac{1}{1-x}\mathrm{d} x$,
%   全部算出来得到$x + (1 - x)\ln (1 - x)$
% \end{proof}

% ~

% \begin{exercise}[分母带$n$型]
%   (1)$\sum\limits_{n = 1}^{\infty}\frac{(-1)^n}{2n}x^{2n} $
%   (2)$\sum\limits_{n = 1}^{\infty}(-1)^{n-1} \frac{2n+1}{n}x^{2n}$
%   (3)注意:$\sum\limits_{n = 1}^{\infty} \frac{x^n}{n+1}$
%   (4)$\sum\limits_{n = 1}^{\infty}(-1)^n \frac{x^{n+1}}{n(n+1)}$
%   (5)$\sum\limits_{n = 1}^{\infty} \frac{x^{n+1}}{n(n+2)}$
% \end{exercise}

% \begin{proof}
%   (1)结果为$- \frac{1}{2} \ln(1 + x^2)$

%   (2)分成$2\sum\limits_{n = 1}^{\infty}(-1)^{n-1}x^{2n} + \sum\limits_{n = 1}^{\infty}(-1)^{n-1}\frac{x^{2n}}{n}$,
%   即$-2 \sum\limits_{n = 1}^{\infty} (-x^2)^n - \sum\limits_{n = 1}^{\infty} \frac{(-x^2)^n}{n}$,
%   前者等于$\frac{2x^2}{1 + x^2}$,
%   后者等于$\ln(1 + x^2)$

%   (3)$\sum\limits_{n = 1}^{\infty} \frac{x^n}{n+1} = x^{-1} \sum\limits_{n = 1}^{\infty} \frac{x^{n+1}}{n+1} = x^{-1} \sum\limits_{n = 1}^{\infty} \frac{x^n}{n} - 1 = - \frac{\ln(1-x)}{x}+1$

%   (4)直接套结论方便:$- \sum\limits_{n = 1}^{\infty} \frac{(-x)^{n+1}}{n(n+1)} = x - (1+x)\ln(1-x)$

%   (5)考虑裂项$\frac{1}{n(n+2)} = \frac{1}{2}\left( \frac{1}{n} - \frac{1}{n+2} \right)$,
%   因此$\sum\limits_{n = 1}^{\infty}\frac{x^{n+1}}{n(n+2)} = \frac{x}{2} \sum\limits_{n = 1}^{\infty} \frac{x^n}{n} - \frac{1}{2x}\sum\limits_{n = 1}^{\infty} \frac{x^{n+2}}{n + 2}$
% \end{proof}


% \subsection{其他幂级数求和:$\arctan x, e^x, (1 + x)^{\alpha}$}

% \begin{theorem}[$\arctan x$展开]
%   $\arctan x = \sum\limits_{n = 0}^{\infty} \frac{(-1)^n}{2n + 1}x^{2n+1}, x \in [-1,1]$
% \end{theorem}

% \begin{proof}
  
% \end{proof}


% \begin{exercise}[计算数项级数]
%   (1)ZJU2020:计算$\sum\limits_{n = 0}^{\infty} \frac{(-1)^n}{3n+1}$
%   (2)计算$\sum\limits_{n = 1}^{\infty}\frac{(-1)^{n+1}}{3n+1}$
%   (3)计算$\sum\limits_{n = 0}^{\infty} \frac{(-1)^n}{3n+2}$
%   (4)计算$\sum\limits_{n = 0}^{\infty} \frac{(-1)^{n+1}}{3n+2}$
% \end{exercise}

% \begin{solution}
%   (1)考虑$f(x) = \sum\limits_{n = 1}^{\infty}\frac{(-1)^n}{3n + 1}x^{3n+1}$,
%   则$f^{\prime}(x) = \sum\limits_{n = 0}^{\infty}(-1)^n x^{3n} = \frac{1}{1+x^3}$,
%   因此
%   \begin{equation*}
%     f(x) = f(0) + \int_0^x f^{\prime}(t)\mathrm{d} t = \frac{1}{3} \ln (x + 1) - \frac{1}{6} \ln (x^2 - x + 1) + \frac{1}{\sqrt{3}} \arctan \frac{2x - 1}{\sqrt{3}} + \frac{\pi}{6 \sqrt{3}}
%   \end{equation*}
%   其中$\frac{1}{1+x^3}$的积分用有理分解算,代入$x = 1$可得到结果为$\frac{1}{3}\ln 2 + \frac{\pi}{3 \sqrt{3}}$
% \end{solution}

% ~

% \begin{theorem}[$(1+x)^{\alpha}$展开]
%   $(1+x)^{\alpha} = 1 + \alpha x + \frac{\alpha(\alpha - 1)}{2!}x^2 + \cdots + \frac{(\alpha)(\alpha - 1)\cdots (\alpha - n + 1)}{n!}x^n + \cdots$
% \end{theorem}







\section{幂级数求和式展开}

\subsection{基本理论与直接展开}

\begin{theorem}[幂级数展开]
  若$f(x)$在$(x_0 - R, x_0 + R)$上存在任意阶导数,
  且$\exists M, N > 0$,当$n > N$时,$\forall x \in (x_0 - R, x_0 + R)$,
  均有$|f^{(n)}(x_0)| \leq M$,
  则
  \begin{equation*}
    f(x) = \sum\limits_{n = 0}^{\infty} \frac{f^{(n)}(x_0)}{n!}(x-x_0)^n, \forall x \in (x_0 - R, x_0 + R)
  \end{equation*}
\end{theorem}

~

\begin{exercise}[幂级数直接展开]
  (1)$\frac{ax + b}{cx + d}$型:$f(x) = \frac{x-3}{x+1}$在$x = 1$处的幂级数展开

  (2)$f(x) = \frac{1}{x(x+1)}$在$x = 1$处幂级数展开

  (3)$f(x) = \frac{1}{1 - 3x + 2x^2}$在$x = 0$处展开

  (4)$f(x) = \frac{x}{1 + x - 2x^2}$在$x = 0$处展开
\end{exercise}

\begin{solution}
  (1)首先$f(x) = 1 - \frac{4}{1+x}$,
  设$t = x - 1$,则$f(x) = g(t) = 1 - \frac{4}{2 + t}$,
  因此$f(x) = 1 - 2 \cdot \frac{1}{1 + \frac{t}{2}} = 1 - 2 \sum\limits_{n = 0}^{\infty}(\frac{-t}{2})^n$,
  因此得到
  \begin{equation*}
    f(x) = 1 - \sum\limits_{n = 0}^{\infty} \left( \frac{-t}{2} \right)^n
  \end{equation*}

  (2)设$t = x- 1$,
  可以得到$f(x) = \sum\limits_{n = 0}^{\infty} (-1)^n \left( 1 - \frac{1}{2^{n+1}} \right)(x-1)^n$
\end{solution}


\subsection{求导与求积}


\begin{exercise}[几个经典的积分与求导法]
  (1)计算$f(x) = \frac{1}{(1+x)^2}$在$x = 0$处的幂级数展开

  (2)计算$f(x) = \arcsin x$在$x = 0$处的幂级数展开

  (3)计算$f(x) = \int_0^x \frac{\sin t}{t}\mathbb{d} t$在$x = 0$处的幂级数展开
\end{exercise}

\begin{solution}
  (1)积分得到$\int_0^x f(t)\mathrm{d} t = - \frac{1}{1+t}\bigg|^x_0 =1 - \frac{1}{1+x} = 1 - \sum\limits_{n = 0}^{\infty}(-x)^n$,
  因此$f(x) = \left( \int_0^x f(t)\mathrm{d}t \right)^{\prime} = \sum\limits_{n = 1}^{\infty}(-1)^{n+1}nx^{n-1}$

  (2)$f^{\prime}(x) = \frac{1}{\sqrt{1 - x^2}}$,
  而
  \begin{equation*}
    (1 - x^2)^{-\frac{1}{2}} = 1 + \sum\limits_{n = 1}^{\infty} \frac{(-\frac{1}{2})(- \frac{3}{2})\cdots (- \frac{1}{2} - n + 1)}{n!}(-x^2)^n = 1+ \sum\limits_{n = 1}^{\infty} \frac{(2n-1)!!}{(2n)!!}x^{2n}
  \end{equation*}
  因此$f(x) = f(0) + \int_0^x f^{\prime}(t)\mathrm{d} t = x + \sum\limits_{n = 1}^{\infty} \frac{(2n-1)!!}{(2n+1)(2n)!!}x^{2n+1}$
\end{solution}

~

\begin{exercise}[$\arctan$经典题目]
  求幂级数展开:

  (1)$f(x) = \arctan x$

  (2)重点(ZJU2021):$g(x) = \arctan \frac{1 - kx}{1 + kx}$

  (3)重点:$h(x) = \arctan \frac{2x}{1 - x^2}$,求导结果很简单
\end{exercise}

\begin{solution}
  (1)$f^{\prime}(x) = \frac{1}{1+x^2} = \sum\limits_{n = 0}^{\infty}(-x^2)^n$,
  因此
  \begin{equation*}
    f(x) = f(0) + \sum\limits_{n = 0}^{\infty} \int_0^x (-1)^n t^{2n}\mathrm{d} t = \sum\limits_{n = 0}^{\infty} (-1)^n \frac{x^{2n+1}}{2n+1}
  \end{equation*}

  (2)$\arctan \frac{1-kx}{1+kx} = \arctan 1 - \arctan kx$,
  因此
  \begin{equation*}
    g(x) = \frac{\pi}{4} - \sum\limits_{n = 0}^{\infty} \frac{(-1)^n k^{2n+1}}{2n+1}x^{2n+1}
  \end{equation*}

  (3)$h^{\prime}(x) = \frac{2}{1+x^2}$(别怕求导!),
  因此$h^{\prime}(x) = 2 \sum\limits_{n = 0}^{\infty} (-x^2)^n$,
  积分得到
  \begin{equation*}
    h(x) = 2 \sum\limits_{n = 0}^{\infty} \frac{(-1)^n}{2n+1}x^{2n+1}
  \end{equation*}
\end{solution}






