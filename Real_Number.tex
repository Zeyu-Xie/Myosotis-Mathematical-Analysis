


\chapter{实数完备性定理}

\begin{definition}[聚点]
  若存在各项互异的收敛数列$\{x_n\} \subseteq S$,
  使得$\lim \limits _{n \rightarrow \infty} x_n = \xi$,
  则$\xi$称为$S$的一个聚点。
\end{definition}

\begin{theorem}[实数七大定理]
  七大定理如下:
  \begin{itemize}
  \item 确界定理:$S$是非空实数集,若$S$有上界,则$S$必有上确界
  \item 单调有界定理
  \item Cauchy收敛准则
  \item 闭区间套定理
  \item 有限开覆盖定理:$S$是实数子集,$H$为开区间族,若$\forall x \in S$都含于$H$至少一个开区间中,则称$H$是$S$开覆盖。若$H$是$[a,b]$的开覆盖,则可从$H$中选出有限个区间覆盖$[a,b]$
  \item 聚点定理:实轴上任一有界无限点集$S$至少有一个聚点
  \item 致密性定理:有界数列必有收敛子列
  \end{itemize}
\end{theorem}

~

下面采取确界原理-单调有界定理-闭区间套定理-有限覆盖定理-聚点定理/致密性定理-Cauchy收敛准则的形式循环证明其等价性:

\begin{enumerate}
\item 确界证明单调有界定理:设$a_n$单增有上界,根据确界原理有上确界,取$a = \sup \{a_n\}$,
  只需证明$a$是$a_n$极限。
  根据上确界定义有$\exists N, a- \epsilon < a_N$,
  再由单增和极限定义显然。
\item 单调有界证明闭区间套:$a_n,b_n$都单调有极限,所以夹住的点存在。
  唯一性用$|\xi - \xi^{\prime}| \leq \lim \limits _{n \rightarrow \infty} (b_n - a_n) = 0$证明
\item 闭区间套证明有限覆盖:反设不成立,则$[a,\frac{a+b}{2}],[\frac{a+b}{2},b]$中至少有一个没有有限覆盖,
  不停分下去,根据闭区间套有$\xi \in [a_n,b_n]$。
  当$n$足够大时,一定存在开覆盖集中的区间包含它,因此与假设矛盾。
\item 有限覆盖证明致密性定理:设$x_n$有界,若无穷项相等,则显然。若相等的只有有限项,
  可知$\exists x_0 \in [a,b]$使得$(x_0 - \delta, x_0 + \delta)$中有无限项,
  否则根据有限开覆盖,总共$x_n$只有有限项,矛盾。
  因此在$(x_0 - \frac{1}{k},x_0 + \frac{1}{k})$中取$x_k$即有$\lim \limits _{k \rightarrow \infty}x_k = x_0$
\item 聚点定理证明Cauchy收敛准则:设$x_n$收敛,$\lim \limits _{n \rightarrow \infty} x_n = a$,
  $\exists N, \forall n,m > N$有$|x_n - a| < \epsilon, |x_m - a| < \epsilon$,
  因此$|a_n - a_m| \leq |a_n - a| + |a - a_m| = 2\epsilon$
\item Cauchy收敛准则证明确界原理:$b$为上界,$a \in S$,不断二分,构造$a_n,b_n$,
  最后$|a_{n+p} - a_n| \leq |b_n - a_n| < \epsilon$,因此有极限$a$,满足$\exists N, a_N + \epsilon > a$。
\end{enumerate}













